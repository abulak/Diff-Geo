\mode*
\mode<all>{\topsection{Wektory związane z krzywą}}

\mode<all>{\midsection{Wektor styczny i normalny}}
\begin{frame}[<+->]
\begin{definicja}
Niech $\alpha\colon (a,b)\to \R^3$ będzie regularną krzywą gładką. Definiujemy \textbf{jednostkowy wektor styczny} do krzywej $\alpha$ w punkcie $t$ jako \[T_\alpha(t)\define\frac{\alpha'(t)}{\|\alpha'(t)\|}.\]
\end{definicja}

\begin{center}
\begin{tikzpicture}[y=0.80pt, x=0.8pt,scale=0.5,yscale=-1, inner sep=0pt, outer sep=0pt]
\begin{scope}[shift={(-103.21718,-652.3486)}]% layer1
  % path1883
  \path[draw=black,fill=black,line join=miter,line cap=butt,line width=0.800pt]
    (662.7078,710.1864) .. controls (645.2037,701.3962) and (626.9213,694.1684) ..
    (608.2231,688.5000) .. controls (589.5249,682.8316) and (570.4139,678.7196) ..
    (551.1186,676.1105) .. controls (540.0689,674.6167) and (528.9779,673.6142) ..
    (517.8668,673.0261) .. controls (500.6069,672.1127) and (483.2866,672.1980) ..
    (465.9972,673.3408) .. controls (448.7078,674.4836) and (431.4508,676.6859) ..
    (414.4144,680.0082) .. controls (399.2183,682.9720) and (384.2001,686.7866) ..
    (369.4447,691.4521) .. controls (351.3715,697.1667) and (333.5173,703.6139) ..
    (315.9428,710.8220) .. controls (298.3683,718.0301) and (281.0766,725.9996) ..
    (264.2123,734.7285) .. controls (253.3037,740.3738) and (242.5650,746.3332) ..
    (232.0302,752.6269) .. controls (210.4765,765.5034) and (189.7930,779.7698) ..
    (169.7027,794.7651) .. controls (149.6124,809.7604) and (130.1097,825.4862) ..
    (110.7984,841.3616) .. controls (105.0485,845.2432) and (103.1497,848.0314) ..
    (103.8589,848.6428) .. controls (104.5680,849.2543) and (107.8758,847.7008) ..
    (112.6387,843.0105) .. controls (131.7961,827.3788) and (151.1280,811.8941) ..
    (171.0241,797.1311) .. controls (190.9201,782.3680) and (211.3858,768.3251) ..
    (232.6933,755.6553) .. controls (243.4823,749.2400) and (254.4829,743.1792) ..
    (265.6568,737.4494) .. controls (282.1373,728.9988) and (299.0102,721.2736) ..
    (316.1354,714.2685) .. controls (333.2607,707.2634) and (350.6356,700.9781) ..
    (368.2001,695.3782) .. controls (383.2438,690.5818) and (398.9685,686.4108) ..
    (415.0627,683.1177) .. controls (432.1722,679.6161) and (449.7015,677.1572) ..
    (467.1191,675.8565) .. controls (484.5367,674.5559) and (501.8379,674.4153) ..
    (518.5322,675.3458) .. controls (529.5002,675.9571) and (540.2091,677.0200) ..
    (550.7362,678.5382) .. controls (569.0301,681.1717) and (586.7344,685.1619) ..
    (604.2196,690.4989) .. controls (621.7047,695.8358) and (638.9769,702.5256) ..
    (656.2516,710.8525) .. controls (658.2912,711.8139) and (660.8876,713.0048) ..
    (663.5305,714.1971) .. controls (666.1735,715.3893) and (668.8631,716.5833) ..
    (671.1079,717.5076) .. controls (673.3527,718.4319) and (675.1544,719.0841) ..
    (676.0284,719.1671) .. controls (676.9023,719.2501) and (676.8519,718.7587) ..
    (675.3620,717.4290) .. controls (671.8545,714.9416) and (667.0095,712.3941) ..
    (662.7078,710.1864) -- cycle;

  % path1904
  \path[draw=black,line join=miter,line cap=butt,line width=0.800pt,o-latex']
    (267.1427,734.5738) -- (419.3410,656.0501);

  % path1906
  \path[draw=black,line join=miter,line cap=butt,line width=0.800pt,o-latex']
    (403.6968,683.1534) -- (531.9686,652.8352);

  % path1908
  \path[draw=black,line join=miter,line cap=butt,line width=0.800pt,o-latex']
    (552.3838,677.2059) -- (714.0883,700.0762);

\end{scope}
\end{tikzpicture}
\end{center}

\end{frame}
%%%%%%next-slide%%%%%
\begin{frame}
 \begin{exercise}
  Niech $v(t)$ i $w(t)$ będą dowolnymi wektorami w $\R^3$, zależnymi od zmiennej $t$. Sprawdzić, że zachodzi \textbf{wz\'or Leibniza} na r\'ożniczkowanie iloczynu skalarnego:
\[\langle v(t),w(t) \rangle'=\langle v(t)',w(t) \rangle+\langle v(t),w(t)' \rangle.\]
 \end{exercise}

\end{frame}


\begin{frame}

\begin{lemat}
Niech $\alpha(t)$ będzie unormowaną krzywą regularną. Wówczas dla każdego $t$ zachodzi \[\langle T(t),T\,'(t)\rangle=0.\]
\end{lemat}

\pause\textcolor{ared}{\textbf{Dowód:}} \\
Zauważmy, że $T(t)$ jest funkcją gładką. Mamy \[1=\|T(t)\|=\langle T(t),T(t)\rangle,\]\pause a więc korzystając powyższego wzoru na różniczkowanie iloczynu skalarnego otrzymujemy:\pause\[0=\|T(t)\|'=\langle T\,'(t),T(t)\rangle+\langle T(t),T\,'(t)\rangle=2\langle T(t),T\,'(t)\rangle.\]
\hfill $\square$

\end{frame}
%%%%%%next-slide%%%%%
\begin{frame}

\begin{definicja}
Załóżmy, że $\alpha\colon (a,b)\to \R^3$ jest krzywą regularną. Dla każdego $t\in (a,b)$ dla którego $\|T\,'(t)\|\neq 0$ definiujemy \textbf{jednostkowy wektor normalny} jako \[N(t)\define \frac{T\,'(t)}{\|T\,'(t)\|}.\]

\pause Jeśli $T(t)$ oraz $N(t)$ są dobrze określone (tj. $T\,'(t)\neq 0$), płaszczyznę rozpiętą przez te dwa wektory nazywamy \textbf{płaszczyzną ściśle styczną}.
\end{definicja}
\medskip
\pause Płaszczyzna ściśle styczna jest w pewnym sensie płaszczyzną najlepiej przybliżającą naszą krzywą, tak jak prosta styczna jest prostą która najlepiej przybliża krzywą $\alpha$.

\end{frame}
%%%%%%next-slide%%%%%
\begin{frame}[<+->]

\begin{lemat}
Niech $\alpha\colon (a,b)\to \R^3$ będzie krzywą regularną. Następujące warunki są równoważne dla każdego $t\in (a,b)$:
\begin{enumerate}
\item $\|T\,'(t)\|\neq 0$,
\item wektory $\alpha'(t)$ oraz $\alpha''(t)$ są liniowo niezależne,
\item $\alpha'(t)\times \alpha''(t)\neq 0$, gdzie $\times$ oznacza iloczyn wektorowy.
\end{enumerate}
\end{lemat}


\end{frame}
%%%%%%next-slide%%%%%
\begin{frame}
\textcolor{ared}{Szkic dowodu:}

\begin{itemize}
\pause\item Implikacje $(2\Leftrightarrow 3)$ wynikają z definicji i własności iloczynu wektorowego.
\pause \item Implikacja $(1\Rightarrow 2)$.
Załóżmy, że istnieje $t_0$ dla którego wektory $\alpha'(t_0)$ i $\alpha''(t_0)$ są liniowo zależne, tj. $\alpha''(t_0)=k\alpha'(t_0)$ dla pewnego $k\in \R$. \pause Pokażemy (bezpośrednim rachunkiem), że wówczas $T(t_0)$ jest wektorem zerowym. \pause Oznaczmy $v(t)\define\|\alpha'(t)\|$ i zauważmy, że $v=\sqrt{(\alpha_1')^2+(\alpha_2')^2+(\alpha_3')^2}$, \pause więc \[v\,'=\frac{2(\alpha_1'\alpha_1''+\alpha_2'\alpha_2''+\alpha_3'\alpha_3'')}{2\sqrt{(\alpha_1')^2+(\alpha_2')^2+(\alpha_3')^2}}=\frac{\langle\alpha',\alpha''\rangle}{v}.\]
\end{itemize}
\end{frame}
%%%%%%next-slide%%%%%
\begin{frame}[<+->]
Mamy wtedy:

\begin{align*}
T\,'(t_0)=&\left(\frac{\alpha'(t_0)}{v(t_0)}\right)'=\frac{\alpha''(t_0) v(t_0)-\alpha'(t_0)v\,'(t_0)}{v^2(t_0)}, \\
\intertext{\pause więc przy podstawieniu $\alpha''(t_0)=k\alpha'(t_0)$ otrzymujemy}
&\pause \frac{k\alpha'(t_0)v(t_0)-\alpha'(t_0)\frac{\langle \alpha'(t_0),k\alpha'(t_0)\rangle}{v(t_0)}}{v^2(t_0)}=\\
&\pause \frac{k\alpha'(t_0)v(t_0)-k\alpha'(t_0)\frac{v(t_0)^2}{v(t_0)}}{v^2(t_0)}=\\
&\pause\frac{k\alpha'(t_0)(v(t_0)-v(t_0))}{v^2(t_0)}=(0,0,0).
\end{align*}

\end{frame}
%%%%%%next-slide%%%%%
\begin{frame}


%Dla uproszczenia zapisu wszystkie funkcje w poniższym rachunku ewaluujemy w punkcie $t_0$.
%
%\begin{align*}
%T\,'=&\left(\frac{\alpha'}{\|\alpha'\|}\right)'=\frac{\alpha'' \|\alpha'\|-\alpha'\|\alpha'\|'}{\|\alpha'\|^2}=\\
%\intertext{co przy założeniu $\alpha''=k\alpha'$ daje}
%=&\frac{k\alpha'\|\alpha'\|-\alpha'\frac{\langle \alpha',k\alpha'\rangle}{\|\alpha'\|}}{\|\alpha'^2\|}=\frac{k\alpha'\|\alpha'\|-k\alpha'\frac{\|\alpha'\|^2}{\|\alpha'\|}}{\|\alpha'^2\|}=\frac{k\alpha'(\|\alpha'\|-\|\alpha\|')}{\|\alpha'\|^2}=0.
%\end{align*}
\begin{itemize}
\item Podobnie udowodnimy implikację $(2\Rightarrow 1)$.

\pause Załóżmy, że istnieje $t_0$ dla którego $\|T\,'\!(t_0)\|=0$. Wtedy sam $T\,'\!(t_0)$ jest wektorem zerowym. Oznaczmy $v(t)\define\|\alpha'(t)\|$. \pause Mamy wtedy
\[0=T\,'(t)\big|_{t=t_0}=\left(\!\frac{\alpha'(t_0)}{v(t_0)}\!\right)'=
\frac{\alpha''(t_0)v(t_0)-\alpha'(t_0)v\,'(t_0)}{v^2(t_0)}.\]
\pause Zatem $\alpha''(t_0)v(t_0)-\alpha'(t_0)v\,'(t_0)=0$, więc albo oba współczynniki (tj. $v(t_0)$ i $v\,'(t_0)$) są zerowe, albo wektory $\alpha'(t_0)$ i $\alpha''(t_0)$ są liniowo zależne. \pause Z regularności krzywej wiemy, że może zachodzić tylko druga sytuacja.
\end{itemize}
\hfill $\square$
\end{frame}
%%%%%%next-slide%%%%%
\mode<all>{\midsection{Wektor binormalny}}
Jeśli dla wszystkich $t\in (a,b)$ z dziedziny krzywa spełnia jeden z powyższych warunków, mamy zdefiniowane w każdym jej punkcie dwa niezależne liniowo (w $\R^3$) wektory. W dodatku są one prostopadłe o długości jednostkowej. Zatem mamy wyznaczony w sposób jednoznaczny kierunek ortogonalny do nich jako prostopadły do płaszczyzny ściśle stycznej (czyli do obu wektorów $T(t)$ i $N(t)$).

\begin{frame}

\begin{definicja}
Niech $\alpha\colon (a,b)\to \R^3$ będzie krzywą regularną. Dla każdego $t\in (a,b)$ takiego, że $\|T\,'(t)\|\neq 0$ definiujemy \textbf{jednostkowy wektor binormalny} jako \[B(t)\define T(t)\times N(t).\]

\pause Płaszczyznę rozpiętą przez wektory $N(t)$ i $B(t)$ nazywamy \textbf{płaszczyzną normalną}, lub \textbf{płaszczyzą prostopadłą} do krzywej.
\end{definicja}


\end{frame}
%%%%%%next-slide%%%%%
\mode<all>{\midsection{Trójnóg Freneta}}
\begin{frame}[<+->]

\begin{definicja}
Układ ortonormalny $\{T(t),N(t),B(t)\}$ nazywać będziemy \textbf{trójnogiem} (lub \textbf{reperem}) Freneta.
\end{definicja}

\begin{center}
\begin{tikzpicture}[y=0.80pt, x=0.8pt,scale=0.5,yscale=-1, inner sep=0pt, outer sep=0pt]
\begin{scope}[shift={(-74.06642,11.12791)}]% layer1
  \begin{scope}% g4657
    % path3997
\definecolor{c1}{HTML}{0000FF}
    \path[fill=c1,opacity=0.320] (74.0664,309.1110) -- (302.2522,-11.1279) --
      (574.4669,23.5572) -- (347.2461,347.1896) -- cycle;

    % path3993
    \path[draw=black,fill=black,line join=miter,line cap=butt,line width=0.800pt]
      (243.5427,-1.2146) .. controls (259.2208,3.3242) and (274.2984,10.0708) ..
      (287.3926,19.9525) .. controls (294.6874,25.4454) and (301.3016,31.9030) ..
      (307.0594,39.1113) .. controls (312.8173,46.3196) and (317.7169,54.2794) ..
      (321.5681,62.7335) .. controls (321.5681,62.7335) and (321.5681,62.7335) ..
      (321.5681,62.7335) .. controls (323.5920,67.1766) and (325.3391,71.7417) ..
      (326.8171,76.3959) .. controls (331.1708,90.1053) and (333.1781,104.5635) ..
      (333.4882,118.9457) .. controls (333.9405,139.9197) and (330.8848,160.8038) ..
      (325.9874,181.1344) .. controls (325.9874,181.1344) and (325.9874,181.1344) ..
      (325.9874,181.1344) .. controls (324.8683,185.7782) and (323.6567,190.3984) ..
      (322.3613,194.9956) .. controls (316.9399,214.2355) and (309.0979,232.7660) ..
      (299.5895,250.3147) .. controls (290.0812,267.8634) and (278.9178,284.4375) ..
      (266.6732,300.1184) .. controls (266.6414,300.1591) and (266.6096,300.1998) ..
      (266.5777,300.2406) .. controls (253.1809,317.3758) and (238.3436,333.4799) ..
      (221.3552,347.2166) .. controls (207.2362,358.6320) and (191.5687,368.3962) ..
      (174.5726,375.0580) .. controls (171.8687,375.7675) and (169.9763,376.5090) ..
      (168.8035,377.1957) .. controls (167.6308,377.8824) and (167.1804,378.5129) ..
      (167.3846,378.9315) .. controls (167.5888,379.3501) and (168.4506,379.5552) ..
      (169.8817,379.3428) .. controls (171.3129,379.1303) and (173.3159,378.4983) ..
      (175.7398,377.2206) .. controls (192.8950,370.5402) and (208.7104,360.7719) ..
      (222.9594,349.3454) .. controls (240.0020,335.6819) and (254.9221,319.6928) ..
      (268.4044,302.6775) .. controls (268.5977,302.4336) and (268.7907,302.1894) ..
      (268.9834,301.9450) .. controls (281.2265,286.4206) and (292.4161,270.0071) ..
      (301.9869,252.6182) .. controls (311.5577,235.2293) and (319.4989,216.8582) ..
      (325.0714,197.7620) .. controls (326.5858,192.5724) and (327.9903,187.2798) ..
      (329.2635,181.9035) .. controls (331.6605,171.7908) and (333.5691,161.3552) ..
      (334.7725,150.7800) .. controls (335.9759,140.2047) and (336.4717,129.4907) ..
      (336.0991,118.8460) .. controls (335.5772,103.8027) and (333.2239,88.8625) ..
      (328.7935,75.0938) .. controls (327.3262,70.5337) and (325.6364,66.1008) ..
      (323.7211,61.8040) .. controls (323.7211,61.8040) and (323.7211,61.8039) ..
      (323.7211,61.8039) .. controls (319.7811,52.9747) and (314.8435,44.7905) ..
      (309.0719,37.3768) .. controls (303.3003,29.9631) and (296.6934,23.3158) ..
      (289.3212,17.5865) .. controls (277.3788,8.3198) and (263.6707,1.5066) ..
      (248.7346,-3.2469) .. controls (246.9995,-3.7785) and (244.7751,-4.3646) ..
      (242.4974,-4.8474) .. controls (240.2197,-5.3301) and (237.8901,-5.7095) ..
      (235.9578,-5.9023) .. controls (234.0255,-6.0950) and (232.4921,-6.1060) ..
      (231.7967,-5.8910) .. controls (231.1014,-5.6760) and (231.2456,-5.2437) ..
      (232.6370,-4.4885) .. controls (235.8000,-3.3110) and (239.9347,-2.3035) ..
      (243.5427,-1.2146) -- cycle;

    % path4042
    \path[draw=red,line join=miter,line cap=butt,line width=0.800pt,o-latex',fill=red]
      (310.5,232) -- (366.0442,100.6789);

    % path3999
    \path[draw=red,line join=miter,line cap=butt,line width=0.800pt,o-latex',fill=red]
      (313,232.37) -- (303.1099,90.2483);

    % path4001
    \path[draw=red,line join=miter,line cap=butt,line width=0.800pt,o-latex',fill=red]
      (317.8,227.8) -- (208.3333,214.3431);

    % text4022
    \path[fill=black] (372.33282,126.04205) node[above right] (text4022) {$T$};

    % text4030
    \path[fill=black] (208.43677,238.67039) node[above right] (text4030) {$N$};

    % text4034
    \path[fill=black] (278.6192,121.25542) node[above right] (text4034) {$B$};

    % path4044
\definecolor{c2}{HTML}{5599FF}
    \path[draw=c2,fill=c2,opacity=0.871,line join=miter,line cap=butt,line
      width=0.800pt] (310.7919,227.2841) .. controls (307.3983,235.1265) and
      (303.6714,242.8093) .. (299.6044,250.3153) .. controls (290.0960,267.8643) and
      (278.9117,284.4466) .. (266.6669,300.1278) .. controls (266.6352,300.1684) and
      (266.6049,300.2122) .. (266.5732,300.2528) .. controls (257.8879,311.3619) and
      (248.6006,322.0309) .. (238.5107,331.9091) -- (241.9170,332.4091) .. controls
      (251.3808,323.0781) and (260.1820,313.0831) .. (268.4170,302.6903) .. controls
      (268.6106,302.4459) and (268.7864,302.1852) .. (268.9795,301.9403) .. controls
      (281.2224,286.4162) and (292.4087,270.0166) .. (301.9795,252.6278) .. controls
      (306.4333,244.5359) and (310.5225,236.2175) .. (314.1982,227.7216) --
      (310.7919,227.2841) -- cycle;

  \end{scope}
\end{scope}
\end{tikzpicture}
\end{center}


\end{frame}
%%%%%%next-slide%%%%%
\begin{uwaga}
\begin{enumerate}
\item Jedyny wybór jaki dokonaliśmy podczas definiowania trójnogu Freneta to kierunek (tj. znak) wektora binormalnego.
\item Definicja $B(t)$ jest uzależniona od tego, że obraz krzywej umieszczony jest w przestrzeni $\R^3$. W wymiarach wyższych jest wiele możliwych wyborów wektora prostopadłego do dwóch danych (innymi słowy: nie ma iloczynu wektorowego)
\end{enumerate}
\end{uwaga}


\mode<all> 

 \mode*
\mode<all>{\topsection{Krzywizna Gaussa II}}
\mode<all>{\midsection{Odwzorowanie Weingartena}}

\begin{frame}
\begin{lemat}
Niech $M\subset \R^3$ będzie powierzchnią gładką, oraz niech $x\colon U\to M$ będzie lokalnym układem współrzędnych wokół punktu $p\in x(U)$. 

\pause Możemy rozważać $\widehat{n}$ jako pole wektorowe na $M$. \pause Wtedy dla każdego wektora $v\in T_p(M)$ pochodna kierunkowa $D\,\widehat{n}(v)$ należy do $T_pM$ (rozważanej abstrakcyjnie jako $2$-wymiarowa podprzestrzeń liniowa w $R^3$).
\end{lemat}

\end{frame}
%%%%%%next-slide%%%%%
\begin{frame}

\textcolor{ared}{\textbf{Dowód:}}\\\pause 
Wektor normalny $\widehat{n}(p)$ ma długość $1$ w każdym punkcie, więc możemy zapisać $\langle\widehat{n},\widehat{n}\rangle=1$ wewnątrz $x(U)$.
\pause \mode<article>{(uwaga: poza tym obszarem zapis $\langle\, , \, \rangle$ nie ma sensu!)}
Wtedy \[0=D\langle\widehat{n},\widehat{n}\rangle(v)=\nabla_v \langle\widehat{n},\widehat{n}\rangle=2\langle\nabla_v \widehat{n},\widehat{n}\rangle=2\langle D\,\widehat{n}(v),\widehat{n}\rangle,\] \pause więc $D\,\widehat{n}(v)$ jest zawsze prostopadły do $\widehat{n}$, zatem musi należeć do $T_pM$.\hfill $\square$

\end{frame}
%%%%%%next-slide%%%%%
\begin{frame}
\begin{definicja}
Przy powyższych oznaczeniach \textbf{odwzorowaniem Weingartena} w punkcie $p$ nazywamy odwzorowanie $L\colon T_pM\to T_pM$ zadane przez \[L(v)\define -D\,\widehat{n}(v)=-\nabla_v\widehat{n}.\]
\end{definicja}
\pause
\begin{lemat}
Odwozorowanie Weingartena $L\colon T_pM\to T_pM$ jest odwzorowaniem liniowym.
\end{lemat}
\pause \textcolor{ared}{\textbf{Dowód:}}\\
Lemat wynika z własności pochodnej kierunkowej (lemat \ref{lem:wl-pochodnej-kierunkowej}).
\hfill $\square$

\end{frame}
%%%%%%next-slide%%%%%

\begin{frame}
\begin{uwaga}
Chociaż do definicji odwzorowania Weingartena używamy lokalnego układu współrzędnych, jednak przy innym wyborze $x\colon U\to M$, odwzorowanie $L$ może się różnić tylko o znak $\pm$.
\end{uwaga}
\end{frame}



\mode<all>{\midsection{Druga forma podstawowa}}
\begin{frame}[<+->]



\begin{definicja}
Niech $M\subset \R^3$ będzie powierzchnią gładką i niech $x\colon U\to M$ będzie lokalnym układem współrzędnych wokół punktu $p\in x(U)$. \textbf{Druga forma podstawowa} w punkcie $p$ to odwzorowanie dwuliniowe $\text{II}_p\colon T_pM\times T_pM\to \R$ indukowane przez odwzorowanie Weingartena $L$, tj. zadane wzorem
\[\text{II}_p(v,w)=\langle L(v),w\rangle,\]
dla wszystkich $v,w$ z przestrzeni stycznej $T_pM$.
\end{definicja}
\begin{uwaga}
Tak jak odwzorowanie Weingartena, druga forma podstawowa jest zdefiniowana z dokładnością do znaku.
\end{uwaga}

\end{frame}
%%%%%%next-slide%%%%%
\begin{frame}[<+->]

\begin{uwaga}[Oznaczenie]
Macierze odwzorowania Weingartena i drugiej formy podstawowej (w standardowej bazie przestrzeni stycznej $x_1,x_2$) oznaczamy odpowiednio  przez 
\begin{align*}
(L_{ij})&=\left[
\begin{array}{cc}
L_{11}&L_{12}\\
L_{21}&L_{22}\\
\end{array}
\right]& (l_{ij})&=\left[
\begin{array}{cc}
l_{11}&l_{12}\\
l_{21}&l_{22}\\
\end{array}
\right]
\end{align*}
\end{uwaga}

\begin{wniosek}
Na podstawie powtórki z algebry liniowej II, mamy \[(l_{ij})=(L_{ij})^t(g_{ij}),\]więc korzystając z własności odwrotności i transpozycji otrzymujemy
\[(L_{ij})=(g_{ij})^{-1}(l_{ij})^t.\]
\end{wniosek}
\end{frame}
%%%%%%next-slide%%%%%
\begin{frame}[<+->]

\mode<article>{Następujący lemat pokazuje sposób na łatwe policzenie współczynników drugiej formy podstawowej $l_{ij}$.}
\begin{lemat}
Niech $M\subset \R^3$ będzie gładką powierzchnią, oraz niech $x\colon U\to M$ będzie lokalnym układem współrzędnych wokół punktu $p\in x(U)$.
\begin{enumerate}
\item [\textbf{1.}](Równania Weingartena) Dla wszystkich indeksów $i,j$, zachodzi 
\[n_i=-L_{1i}x_1-L_{2i}x_2.\]
\item [\textbf{2.}]Dla wszystkich indeksów $i,j$, współczynniki macierzy drugiej formy podstawowej są równe
\[l_{ij}=-\langle n_i,x_j\rangle=\langle n,x_{ij}\rangle,\]
gdzie $x_{ij}$ jest oznaczeniem drugiej pochodnej cząstkowej względem zmienych $i$-tej i $j$-tej.
\end{enumerate}
\end{lemat}
\end{frame}
%%%%%%next-slide%%%%%
\begin{frame}

\textcolor{ared}{\textbf{Dowód:}}\\\pause 
\begin{enumerate}
\item [(\textbf{1.})]Mamy następujący ciąg równości:
\[n_i=\frac{\partial(\widehat{n}\circ x)}{\partial u_i}=\nabla_{x_i}\widehat{n}=-L(x_i)=-L_{1i}x_1-L_{2i}x_2,\]
gdzie $x=x(u_1,u_2)$ ($u_i$ są zmiennymi lokalnego układu współrzędnych $x$).
\pause \item [(\textbf{2.})]
Mamy \[l_{ij}=\text{II}(x_i,x_j)=\langle L(x_i),x_j\rangle\stackrel{*}{=}-\langle\nabla_{x_i}n,x_j\rangle=-\langle n_i,x_j\rangle,\]
co dowodzi pierwszej równości w punkcie 2. (równość $*$ wynika z dowodu pierwszej części.) \pause Aby udowodnić drugą równość, skorzystamy z tego, że $\langle n, x_i\rangle=0$. Mamy 
\pause \[0=\frac{\partial \langle n,x_j\rangle}{\partial u_i}=\langle n_i,x_j\rangle+\langle n,x_{ij}\rangle,\]skąd natychmiast wynika druga równość. \hfill $\square$
\end{enumerate}

\end{frame}
%%%%%%next-slide%%%%%
\begin{frame}

\begin{lemat}
\begin{itemize}
\pause \item Druga forma podstawowa $\text{II}$ jest symetryczna.
\pause \item Odwzorowanie Weingartena $L$ jest samosprzężone.
\end{itemize}
\end{lemat}
\textcolor{ared}{\textbf{Dowód:}}\\\pause 
Symetryczność macierzy $(l_{ij})$ wynika z poprzedniego lematu i równości $x_{12}=x_{21}$. \pause Druga teza wynika wtedy z powiązań macierzy symetrycznej z samosprzężeniem odwzorowania przez nią indukowanego (lemat \ref{lem:alg-lin-2} cytowany podczas powtórki z algebry liniowej II).
\hfill $\square$

\end{frame}
%%%%%%next-slide%%%%%
\begin{frame}[<+->]

\begin{uwaga}
Z powyższych rozważań wcale nie wynika, że macierz odwzorowania Weingartena $(L_{ij})$ jest symetryczna. Jeśli baza przestrzeni stycznej $\{x_1, x_2\}$ nie będzie ortonormalna w punkcie $p$, wtedy najczęściej $L_{ij}(p)$ nie będzie macierzą symetryczną. (ogólniej: nie możemy wtedy zastosować do niej lematu \ref{lem:alg-lin-2}). 
\end{uwaga}

\begin{uwaga}
Wiedząc, że $l_{ij}$ jest symetryczna, możemy przepisać uzyskaną wcześniej równość
\[(L_{ij})=(g_{ij})^{-1}(l_{ij}).\]
\end{uwaga}

\end{frame}
%%%%%%next-slide%%%%%
\mode<all>{\midsection{Krzywizna Gaussa oraz krzywizna średnia}}
\begin{frame}
\mode<article>{Jak pamiętamy intuicyjna definicja krzywizny wokół punktu $p$ kazała nam porównywać pole na powierzchni z polem zakreślonym przez wektor normalny. Ponieważ odwzorowanie Weingartena charakteryzuje lokalne zmiany wektora normalnego, może się nadawać do definicji krzywizny. }

Niezmiennikami numerycznymi macierzy $2\times2$ są wyznacznik i ślad. Co więcej, są to niezmienniki odpowiadajacego danej macierzy odwzorowania liniowego (tj. są te same dla macierzy sprzężonych), dlatego właśnie je użyjemy w poniższych definicjach.

\pause \begin{definicja}
Niech $M\subset \R^3$ będzie powierzchnią gładką i niech $L$ będzie oznaczało odwzorowanie Weingartena. \pause Zdefiniujmy dwie funkcje skalarne $K\colon M\to \R$, $H\colon M\to \R$ nastepująco
\begin{align*}
K(p)&=\det L(p) & H(p)=\frac{1}{2}\text{tr}\, L(p).
\end{align*}
Nazywamy je odpowiednio \textbf{krzywizną Gaussa} i \textbf{krzywizną średnią}.
\end{definicja}

\end{frame}
%%%%%%next-slide%%%%%
\begin{frame}

\begin{lemat}
Krzywizna Gaussa i krzywizna średnia nie zależą od wyboru macierzy reprezentującej odwzorowanie Weingartena, tj. nie zależą od wyboru bazy przestrzeni stycznej $T_pM$.
\end{lemat}

\pause \textcolor{ared}{\textbf{Dowód:}}\\
Dowód wynika z odpowiedniego przedstawienia wyznacznika (jako iloczynu wartości własnych) i śladu (jako ich sumy)  cytowanego w powtórce z algebry liniowej II (Lemat \ref{lem:alg-lin-2-eigen}). \hfill $\square$

\end{frame}
%%%%%%next-slide%%%%%
\begin{frame}

\begin{lemat}
Niech $M\subset \R^3$ będzie powierzchnią gładką, oraz niech $x\colon U\to M$ będzie lokalnym układem współrzędnych wokół punktu $p\in x(U)$. \pause Wtedy
\begin{align*}
K(p)=&\frac{\det (l_{ij})}{\det (g_{ij})},\quad\text{oraz}\quad& H(p)=&\frac{g_{11}l_{22}-2g_{12}l_{12}+g_{22}l_{11}}{2 \det (g_{ij})}
\end{align*}
% \begin{align*}
% K(p)=&\frac{\det (l_{ij})}{\det (g_{ij})},\\\intertext{oraz}
% H(p)=&\frac{g_{11}l_{22}-2g_{12}l_{12}+g_{22}l_{11}}{2 \det (g_{ij})}
% \end{align*}
\end{lemat}

\pause \textcolor{ared}{\textbf{Dowód:}}\\
Dowody tych równości wynikają z równości
$(L_{ij})=(g_{ij})^{-1}(l_{ij})$, oraz z własności multiplikatywnych wyznacznika i śladu macierzy. Pozostawiamy je do sprawdzenia jako zadanie domowe.
\hfill $\square$

\end{frame}
%%%%%%next-slide%%%%%
\mode<all>{\lowsection{Podsumowanie}}
\begin{frame}{Podsumowanie}

Aby obliczyć krzywizny (średnią i Gaussa) powierzchni potrzebujemy następujące wielkości:
\begin{align*}
g_{11}=&\langle x_1,x_1\rangle, & g_{12}=g_{21}=&\langle x_1,x_2\rangle, & g_{22}=&\langle x_2,x_2\rangle, \\
\intertext{\[n=\frac{x_1\times x_2}{||x_1\times x_2||}=\frac{x_1\times x_2}{\sqrt{\det (g_{ij})}},\]\vspace*{-0.2in}}
l_{11}=&\langle n_1,x_1\rangle, & l_{12}=&\langle n_2,x_1\rangle, & l_{22}=&\langle n_2,x_2\rangle, \\
\intertext{\[K(p)=\frac{\det (l_{ij})}{\det (g_{ij})},\qquad  H(p)=\frac{g_{11}l_{22}-2g_{12}l_{12}+g_{22}l_{11}}{2 \det (g_{ij})}.\]\vspace*{-0.2in}}
\end{align*}

\end{frame}
%%%%%%next-slide%%%%%
\mode<all>{\midsection{Agitacja na rzecz zgodności definicji}}
\begin{frame}[<+->]{Agitacja na rzecz zgodności definicji}

\mode<article>{Nie podaliśmy precyzyjnej definicji orginalnej krzywizny Gaussa, więc trudno mówić o dowodzie równoważności naszej (precyzyjnej) definicji. Niemniej jednak postaramy się zmotywować tę równoważność. Jak zwykle niech $M\subset \R^3$ będzie gładką powierzchnią i niech $x\colon U\to M$ będzie lokalnym układem współrzędnych wokół $p\in M$. Oznaczmy przez $\overline{p}=x^{-1}(p)$.}

\mode<presentation>{Niech $M\subset \R^3$ będzie gładką powierzchnią i niech $x\colon U\to M$ będzie lokalnym układem współrzędnych wokół $p\in M$. Oznaczmy przez $\overline{p}=x^{-1}(p)$.}

\pause Przypomnijmy orginalną definicję Gaussa krzywizny i zastąpmy pola przez odpowiednie całki:
\begin{multline*}
K(p)=\lim_{T\to \{p\}}\frac{A(\widehat{n}(T))}{A(T)}\pause =\frac{\iint_{x^{-1}(T)}\langle n_1\times n_2, n\rangle ds\,dt}{\iint_{x^{-1}(T)}\langle x_1\times x_2, n\rangle ds\,dt}=\\=\frac{\iint_{x^{-1}(T)}\langle n_1\times n_2, n\rangle ds\,dt}{\iint_{x^{-1}(T)}\sqrt{\det(g_{ij}) } ds\,dt}. 
\end{multline*}
\end{frame}
%%%%%%next-slide%%%%%
\begin{frame}[<+->]
Teraz użyjemy twierdzenia o wartości średniej które mówi, że dla każdego takiego zbioru $T$ muszą istnieć takie punkty $a_T,b_T\in x^{-1}(T)$, że cała całka wyraża się jako wartość funkcji podcałkowej w tych punktach:
\pause \begin{align*}
\iint_{x^{-1}(T)}\langle n_1\times n_2, n\rangle ds\,dt=&\langle n_1(a_T)\times n_2(a_T),n(a_T)\rangle A(x^{-1}(T)),\\
\iint_{x^{-1}(T)}\sqrt{\det(g_{ij}) } ds\,dt=& \sqrt{\det(g_{ij}(b_T))}A(x^{-1}(T)).
\end{align*}

\end{frame}
%%%%%%next-slide%%%%%
\begin{frame}[<+->]


Zauważmy, że skoro $T\to \{p\}$, więc $a_T\to \overline{p}$ oraz $b_T\to \overline{p}$. \pause Mamy więc
\begin{multline*}
\lim_{T\to \{p\}}\frac{A(\widehat{n}(T))}{A(T)}=\lim_{T\to \{p\}}\frac{\langle n_1(a_T)\times n_2(a_T),n(a_T)\rangle A(x^{-1}(T))}{\sqrt{\det(g_{ij}(b_T))}A(x^{-1}(T))}=\\
=\frac{\langle n_1(\overline{p})\times n_2(\overline{p}),n(\overline{p})\rangle}{\sqrt{\det(g_{ij}(\overline{p}))}}.
\end{multline*}
\pause
Z równań Weingartena na pochodne wektora normalnego ($n_i=-L_{1i}x_1-L_{2i}x_2$) otrzymujemy
\begin{multline*}
 n_1\times n_2=\big(-(L_{11}x_1+L_{21}x_2)\big)\times\big(-(L_{21}x_1+L_{22}x_2)\big)=\\=(x_1\times x_2)(L_{11}L_{22}-L_{21}L_{22})=K(p)(x_1\times x_2).
\end{multline*}

\end{frame}
%%%%%%next-slide%%%%%
\begin{frame}[<+->]

Podstawiając wyliczony iloczyn wektorowy oraz korzystając z definicji wektora normalnego mamy 
\begin{multline*}
\langle n_1(\overline{p})\times n_2(\overline{p}) ,n(\overline{p})\rangle=\pause K(p)\left\langle x_1(\overline{p})\times x_2(\overline{p}),\frac{x_1(\overline{p})\times x_2(\overline{p})}{\sqrt{\det (g_{ij}(\overline{p}))}}\right\rangle=\\\pause
=\frac{K(p)}{\sqrt{\det (g_{ij}(\overline{p})))}}||x_1\times x_2||^2=\pause 4K(p)\sqrt{\det (g_{ij}(\overline{p})))},
\end{multline*}
\pause zatem ostatecznie 
\[\lim_{T\to \{p\}}\frac{A(\widehat{n}(T))}{A(T)}=\frac{K(p)\sqrt{\det (g_{ij}(\overline{p})))}}{\sqrt{\det (g_{ij}(\overline{p})))}}=K(p).\]

\end{frame}
\mode<all>
\documentclass[a4paper,11pt]{article}
\usepackage[utf8]{inputenc}
\usepackage{polski}
\usepackage{mathtools}
% \usepackage{geometry}
\usepackage{hyperref}
\usepackage{amsmath}
\usepackage{amsthm}
\usepackage{amsfonts}
\usepackage{amssymb}
\usepackage{fourier}
\usepackage[T1]{fontenc}


\setcounter{secnumdepth}{-1 }
%opening
\title{Zadania}
\author{Marek Kaluba\\Elementy Topologii i Geometrii Różniczkowej}

\theoremstyle{definition}\newtheorem{exercise}{Zadanie}
\theoremstyle{definition}\newtheorem{remark}{Uwaga}


\begin{document}
\section{Krzywe w $\mathbb{R}^3$}

\subsection{Zadania elementarne}

\begin{exercise}
 Obliczyć długość następujących wektor\'ow w $\mathbb{R}^3$:
\begin{itemize}
 \item $\left(1,1,1\right)$
 \item $\left(3,\pi,2\sqrt{3}\right)$
 \item $\left(7\frac{1}{2},e,3\sqrt{2}\right)$
 \item $\left(\cos \phi, \sin \phi \cos \psi, \sin\phi\sin\psi\right)$ dla 
ustalonych 
kąt\'ow $\phi$ i $\psi$.
 
\end{itemize}
\end{exercise}

\begin{exercise}
Znaleźć postać parametryczną prostej  w $\mathbb{R}^3$ przechodzącej
przez punkty $\left(3,\pi,2\sqrt{3}\right)$ oraz 
$\left(7\frac{1}{2},e,3\sqrt{2}\right)$.
\end{exercise}

\begin{exercise}
Pokazać, że $\alpha$ jest prostą wtedy i tylko wtedy gdy $\alpha''\equiv 0$.
\end{exercise}

\begin{exercise}
Udowodnić, że żadne 4 różne punkty leżące na krzywej $\left(t,t^2,t^3\right)$ 
nie leżą 
na jednej płaszczyźnie.
\end{exercise}

\begin{exercise}
Udowodnić, że rzut ortogonaly krzywej $\alpha$ na dowolną oś ma długość co 
najwyżej równą długości wyjściowej krzywej.
\end{exercise}

\subsection{Znajdowanie parametryzacji}

\begin{exercise}
Rozważmy okrąg o promieniu $r$ i środku w punkcie $(0,r)$. Niech $P$ będzie 
punktem na okręgu o wsp\'ołrzędnych w $\mathbb{R}^2$ r\'ownych $(0,0)$. Okrąg 
ten zaczyna się toczyć po prostej (w prawo bądź w lewo). Wyznaczyć r\'ownanie 
krzywej po kt\'orej porusza się punkt $P$ (jest to tzw. \textit{cykloida}).
\end{exercise}

\begin{exercise}
Okrąg o promieniu $r$ toczy się wewnątrz okręgu o promieniu $nr$. Wyznaczyć 
r\'ownanie krzywej po kt\'orej porusza się punkt $P$ będący początkowym punktem 
styczności obu okręg\'ow. (jest to tzw. \textit{asteroida}).
\end{exercise}

\begin{exercise}
 Sprawdzić, że asteroida dla $n=4$ może być opisana równaniami:
\[\alpha(t)=\left(a \cos^3{t},b\cos^3{t}\right)\text{ (r\'ownanie 
parametryczne)},\] 
\[x^{2/3}+y^{2/3}=a^{2/3}\text{ (r\'ownanie analityczne)}.\]
\end{exercise}

\begin{exercise}
Niech $l(t)=(t,at+b)$ będzie krzywą na płaszczyźnie przechodzącą przez punkt 
$(-1,0)$. Sparametryzować jedną z gałęzi hiperboli zadanej wzorem $x^2-y^2=1$ 
jako parametr wybierając współczynnik $b$.
\end{exercise}

\begin{exercise}
Niech $l(t)=(t,at+b)$ będzie krzywą na płaszczyźnie przechodzącą przez punkt 
$(-1,0)$. Sparametryzować krzywą zadaną równaniem $x^2+y^2=1$ jako parametr 
wybierając współczynnik $b$ (uwaga: parametryzacja nie obejmuje punktu 
$(-1,0)$!)
\end{exercise}



\begin{exercise}[fizyczne]
Zbadać kształt krzywej mostu wiszącego, tj. krzywej $\alpha(t)$, kt\'orej 
``ciężar'' rozłożony jest jednorodnie wzdłuż osi $OX$. 

Korzystając z powyższego rysunku sprowadzić to zadanie do rozwiązania układu 
równań:
\begin{align*}
\alpha'(t)\cos\theta&=T\\
\alpha'(t)\sin\theta&=Ct
\end{align*}
gdzie 
\begin{itemize}
 \item $\theta$ to kąt między wektorem $\alpha'(t)$ a osią $OX$, 
 \item $T$ jest pewną stałą (jaka jest jej interpretacja fizyczna?),
 \item $C$ jest pewną stałą (jaka jest jej interpretacja fizyczna?).
\end{itemize}
Następnie rozwiązać układ pamiętając o tym, że $\alpha'(t)=\tg\theta$.
\end{exercise}

\begin{exercise}[fizyczne]
Sprowadzić powyższe zadanie do r\'ownania krzywej opisującego rzut ukośny 
(pocisk wystrzelony pod kątem $\theta$ w jednorodnym polu grawitacyjnym).
\end{exercise}

\subsection{Reparametryzacja}

\begin{exercise}
Pokazać, że jeśli krzywe (regularne) $\alpha$ i $\overline{\alpha}$ mają ten 
sam kształ (tj. wykres w $R^3$), w\'owczas jedna z nich jest 
reparametryzacją (gładką) drugiej.
\end{exercise}

\begin{exercise}
Omówić dowód istnienia parametryzacji unormowanej.
\begin{enumerate}
 \item Niech $s(t)=\int_{a}^t|\alpha'(x)|dx$.
\item $s(t)$ jest funkcją ściśle rosnącą (bo krzywa jest unormowana), więc 
posiada funkcję odwrotną:\[t(s)=s^{-1}(t),\] która jest szukaną 
reparametryzacją krzywej:
\item $\alpha(t(s))$ jest krzywą unormowaną.
\end{enumerate}
\end{exercise}

\begin{exercise}
Wyznaczyć parametryzację unormowaną dla
\begin{itemize}
\item okręgu o promieniu $r$,
\item linia śrubowa (helisy) o promieniu $a$ i współczynniku nachylenia $b$
\item krzywej zadanej przez \[\alpha(t)=\left(e^t,e^{-t},\sqrt{2}t\right).\]
\end{itemize}
\end{exercise}

% \begin{exercise}
% Grupa wymyśla własne przykłady, pr\'obując znaleźć parametrtzację unormowaną.
% \end{exercise}

\subsection{Niezmienniki krzywych, Tr\'ojn\'og Freneta}

\begin{exercise}
Obliczyć wektor styczny i normalny do okręgu o promieniu $r$ i środku w 
punkcie $(0,0)$.
\end{exercise}

\begin{exercise}
Znaleźć wektor styczny i jego długość:
\[\alpha(t)=\left(\frac{\lambda}{2}\left(\frac{1}{t}+2t+t^3\right),\frac{\lambda
}{2}\left(\ln 
{\frac{1}{t}}+t^2+\frac{3}{4}t^4\right)-\frac{7}{8}\lambda\right)\]
\end{exercise}

\begin{exercise}
 Niech $\alpha(t)$ będzie krzywą w $\mathbb{R}^2$ zadaną przez wykres funkcji 
$f\colon 
\mathbb{R}\to \mathbb{R}$. Znaleźć wektory styczny i normalny do $\alpha$. 
Pokazać, że krzywizna $\alpha$ jest r\'owna
\[\kappa=\frac{\left|f''\right|}{\left(1+\left(f'\right)^2\right)^{3/2}}.\] 
\end{exercise}

\begin{exercise}
Znaleźć krzywą płaską $\alpha$ (o parametryzacji unormowanej), której krzywizna 
wynosi \[\kappa_\alpha(s)=\frac{1}{s}.\]
\footnotesize{(Podpowiedź: 
$\alpha(s)=\left(\int_0^s\sin(\vartheta(u))\,du,\int_0^s\cos(\vartheta(u))\,
du\right)$. }
\end{exercise}

\begin{exercise}
Dla krzywych unormowanych sprawdzić 
\begin{itemize}
 \item wz\'or Freneta na $T'$ 
 \item wz\'or Freneta na $B'$ 
\end{itemize}

\end{exercise}

\begin{exercise}
Wyznaczyć Tr\'ojn\'og Freneta oraz torsję i krzywiznę dla następujących 
krzywych.
\begin{itemize}
 \item linia śrubowa
\item 
\[\alpha(t)=\left(t,t^2,t^3\right),\]
\item 
\[\alpha(t)=\left(\frac{t^2}{2},\sqrt{2}\frac{t^3}{3},\frac{t^4}{4}\right)\]
\item 
\[\beta(s)=\left(\frac{(1+s)^{\frac{3}{2}}}{3},\frac{(1-s)^{\frac{3}
{ 2}}}{3},
\frac{s}{\sqrt{2}}\right),\]
\item \[\alpha(t)=\left(2\ln t,2t,\frac{t^2}{2}\right)\]

\end{itemize}

\end{exercise}

\begin{exercise}
Dla krzywej regularnej $\alpha$ (niekoniecznie unormowanej) wyprowadzić 
następujące wzory:
\begin{align*}
 T=& \frac{\alpha'}{\|\alpha'\|},& 
B=& \frac{\alpha'\times \alpha''}{\|\alpha'\times 
\alpha''\|}, & N=& B\times T,
\end{align*}
\begin{align*}
\kappa=&\frac{\|\alpha'\times \alpha''\|}{\|\alpha'\|^{3}}, & 
\tau=&\frac{\langle \alpha'\times 
\alpha'',\alpha'''\rangle}{\|\alpha'\times \alpha''\|}.
\end{align*}
\end{exercise}

\begin{exercise}
Niech \[\beta(t)=\int_0^t B_\alpha (s)\,ds\] dla pewnej krzywej unormowanej 
$\alpha$. Wyrazić trójnóg Freneta dla krzywej $\beta$, t.j.
$\left(T_\beta,B_\beta,N_\beta\right)$ oraz krzywiznę $\kappa_\beta$ i 
torsję 
$\tau_\beta$ przy pomocy tychże niezmienników krzywej $\alpha$.
\end{exercise}


\begin{exercise}
Więcej o iloczynie wektorowym: Wektor Darboux.\\
Niech $\omega$ będzie takim wektorem, że 
\begin{align*}
T'&=\omega\times T\\
N'&=\omega\times N\\
B'&=\omega\times B
\end{align*}
Pokazać, że $\omega=\tau T+\kappa B$

\end{exercise}


\subsection{Ewolwenty i ewoluty}
\begin{exercise}
Niech $\alpha$ będzie krzywą regularną. \textit{Ewolwenta} (lub 
\textit{rozwijająca}) krzywej $\alpha$ ma następującą interpretację 
geometryczną.

\begin{quote}
Wyobraźmy sobie, że punkt $A$ porusza się po krzywej $\alpha(t)$ ciągnąc za 
sobą punkt $B$ na linie kt\'orej długość jest r\'owna długości drogi kt\'orą 
przebiegł punkt $A$. Krzywą po kt\'orej porusza się punkt $B$ nazywamy 
ewolwentą krzywej $\alpha(t)$ i oznaczamy $\mathcal{E}(\alpha)(t)$.
\end{quote}

Korzystając z tej interpretacji znajdź wz\'or (zależny od $\alpha$) kt\'orym 
wyraża się $\mathcal{E}(\alpha)(t)$.
\end{exercise}

\begin{exercise}
Niech $\alpha$ będzie krzywą regularną. \textit{Ewoluta} krzywej $\alpha$ ma 
następującą interpretację geometryczną.

\begin{quote}
W każdym punkcie krzywej $\alpha$ narysujmy okrąg ściśle styczny do 
$\alpha(t)$, tj. okrąg styczny do $\alpha$, leżący w płaszczyźnie rozpiętej 
przez wektory $T_\alpha(t)$ i $N_\alpha(t)$, o krzywiźnie 
r\'ownej odwrotności krzywizny w danym punkcie ($\frac{1}{\kappa_{\alpha} 
(t)}$). Środki tych okręg\'ow dla zmieniającego się $t$ utworzą krzywą
kt\'orą nazywamy ewolutą krzywej $\alpha(t)$ i oznaczamy $E(\alpha)(t)$.
\end{quote}

Korzystając z tej interpretacji znajdź wz\'or (zależny od $\alpha$) kt\'orym 
wyraża się $E(\alpha)(t)$.
\end{exercise}


\begin{exercise}
Pokaż, że ewolwenta ($\mathcal{E}$) ewoluty jest równa wyjściowej krzywej, tj.
\[\mathcal{E}(E(\alpha))(t)=\alpha(t).\] 
\end{exercise}

\begin{exercise}
Pokaż, że ewoluta ewolwenty jest równa wyjściowej krzywej, tj.
\[E(\mathcal{E}(\alpha))(t)=\alpha(t).\] 
\end{exercise}

\subsection{Zadania r\'ożne}

\begin{exercise}
Pokaż, że jeśli wszystkie proste styczne do krzywej $\alpha$ zawierają jeden 
punkt, to krzywa ta jest prostą (odcinkiem).
\end{exercise}

\begin{exercise}
Pokaż, że jeśli wszystkie proste normalne do krzywej $\alpha$ zawierają jeden 
punkt, to krzywa ta jest okręgiem.
\end{exercise}

\begin{exercise}
 Pokaż, że jeśli wszystkie płaszczyzny normalne do krzywej zawierają jeden 
punkt, to krzywa ta jest krzywą sferyczną (i.e. leży na powierzchni sfery).
\end{exercise}

\begin{exercise}
Udowodnij, że jeśli wszystkie wektory binormalne do krzywej $\alpha$ są 
równoległe, to $\alpha$ jest krzywą płaską.
\end{exercise}

\begin{exercise} 
Pokaż że krzywa o stałej torsji $\tau$ i krzywiźnie $\kappa$ jest linią śrubową 
postaci
\[(a \sin{x},a\cos{x},b x).\]
Wyrazić $a$ i $b$ w terminach $\tau$ i $\kappa$.
\end{exercise}

\begin{exercise}
Obliczyć paramatryzację unormowaną dla elipsy. Dlaczego są z tym problemy?
\end{exercise}

\begin{exercise}
Pokazać, że krzywa zadana wzorem
\[\alpha(t)=\left\{
\begin{array}{lc}
\left(\tg\left(\frac{\pi}{4}t\right),\sin\left(\pi t\right),t 
\sin\left(\frac{\pi} {t}\right)\right) & t\in (0,1]\\
(0,0,0) & t=0
\end{array}\right.\]
ma nieskończoną długość.

\end{exercise}

\begin{exercise}
Fred Flinstone ma samochód o kołach będącymi kwadratami o przekątnej równej $2$. W jaki sposób powinien zaprojektować drogę, żeby jechać po niej bez wstrząsów? (tj. podczas toczenia się, środek kwadratu ma mieć współrzędną $y$ równą stale $1$).
\end{exercise}

\begin{exercise}
Pokazać, że jeśli $|\alpha(t)|>R$ dla wszystkich $t\in (-\varepsilon, 
\epsilon)$, $t\neq 0$, oraz $|\alpha(0)|=R$, to 
\[\kappa(0)\leqslant \frac{1}{R}.\]
Czy da się udowodnić, że $\kappa(0)<\frac{1}{R}$?
\end{exercise}

\begin{exercise}[Wektor Darboux]
Zamiast trójnogu Freneta można dla danej krzywej (o prędkości jednostkowej) $\alpha\colon[a,b]\to \mathbb{R}^3$ określić układ ${T, U, V }$, biorąc jako $T$ wektor styczny do $\alpha$, żądając, aby $U$ było dowolnym jednostkowym polem wektorowym wzdłuż $\alpha$ takim, że $T\cdot U= 0$, tzn. odwzorowanie $U\colon[a, b] \to \mathbb{R}^3$ przyporządkowuje każdemu $t \in [a, b]$ wektor jednostkowy $U(t)$ prostopadły do wektora $T(t)$. Niech $V = T \times U$. Pokaż, że naturalne związki (czyli ``wzory Freneta'') dla tego układu mają postać:
\begin{align*}
T =& \omega_3 U - \omega_2V \\
U =& -\omega_3T + \omega_1 V\\
V =& \omega_2 T - \omega_1U
\end{align*}
gdzie współczynniki $\omega_1$, $\omega_2$ i $\omega_3$ są rzeczywiste. Ponadto 
pokaż, że wektor $\omega$ (zwany wektorem Darboux) spełniający zależności $T = 
\omega\times T$ , $U = \omega\times  U$ oraz $V = \omega\times  V$
jest postaci \[\omega = \omega_1T + \omega_2U + \omega_3 V \]
\end{exercise}

\begin{exercise}
Dwie krzywe $\alpha$ i $\beta$ nazywamy \textit{parą Bertranda} jeśli dla 
każdego $t$, prosta normalna do $\alpha$ przechodząca przez punkt $\alpha(t)$ 
jest r\'owna prostej normalnej do $\beta$ przechodzącej przez punkt $\beta(t)$.
Pokazać, że zachodzą następujące własności.

\begin{itemize}
 \item Jeśli $\alpha$ ma parametryzację unormowaną, w\'owczas $\beta=\alpha+ 
cN_\alpha$ dla pewnej stałej $c$. 
\item Co więcej, kąt między $T_{\alpha}$ i $T_\beta$ jest stały.
\item Zał\'ożmy, że $\alpha$ jest niepłaską krzywą unormowaną. Pokazać, że 
$\alpha$ ma parę Bertranda wtedy i tylko wtedy, gdy istnieją stałe $c_1$ i 
$c_2$ spełniające $c_1 \kappa_{\alpha} (t)+c_2 \tau_\alpha(t)=1$.

\item Zał\'ożmy, że istnieje więcej niż jedna krzywa $\beta$ kt\'ora stanowi 
parę Bertranda dla $\alpha$. Pokazać, że w\'owczas istnieje ich nieskończenie 
wiele. Pokazać, że taka sytuacja zachodzi wtedy i tylko wtedy, gdy $\alpha$ 
jest linią śrubową.
\end{itemize}



\end{exercise}

\subsection{Twierdzenie klasyfikacyjne dla krzywych (Wykład 4)}

\begin{exercise}
 Niech $A\in SO(3)$ będzie macierzą o kolumnach ortonormalnych. Pokazać, że dla 
dowolnej krzywej $\alpha$, krzywa
\[\gamma(t) = A\cdot \alpha(t)\]
ma te same niezmienniki (tj. $(T,N,B,\kappa,\tau)$).
\end{exercise}

\begin{exercise}
Znaleźć interpretację umożliwiającą zastosowanie Twierdzenia Picarda
(Twierdzenie 4.3) do dowodu twierdzenia klasyfikacyjnego. 
\end{exercise}

\begin{exercise}
Sformułować układ równań wiążących pochodne funkcji $p'_{i,j}(t)$ z funkcjami 
$\{p_{i,j}(t)\}$ oraz $\kappa(t)$ i $\tau(t)$.
\end{exercise}

\begin{exercise}
Pokazać, że otrzymany w dowodzie wektor $X_3(t)$ ma ten sam zwrot co 
$B_\alpha(t)$ dla wszystkich $t$.
\end{exercise}


% \begin{exercise}
% Cytując odpowiednie wyniki z teorii równań różniczkowych zwyczajnych i 
% korzystając z podręczników, przedstawić pełny dowód twierdzenia 
% klasyfikacyjnego dla krzywych.
% \end{exercise}

\section{Powierzchnie}

\subsection{Parametryzacje powierzchni}
\begin{exercise}
Zastanowić się jak pokryć całą powierzchnię sfery jednym płatem 
powierzchniowym. Gdzie pojawiają się problemy?
\end{exercise}

\begin{exercise}(Projekcja stereograficzna)
 Rozważmy sferę $S^2\subset \mathbb{R}^3$ o promieniu $1$ i środku w punkcie 
$(0,0,0)$. Niech 
$l_{(x,y)}$ oznacza prostą w $\mathbb{R}^3$ przechodzącą przez punkt $(x,y,0)$ 
oraz przez punkt $(0,0,1)$. 
\begin{itemize}
 \item Pokazać, że każda taka prosta przecina $S^2$ w dokładnie dw\'och 
punktach: $(0,0,1)$ oraz $(a,b,c)$. Znaleźć wsp\'ołrzędne $a,b,c$ w terminach 
$x$ i $y$.
\item Pokazać, że przyporządkowanie
\[(x,y)\mapsto (a,b,c)\]
jest parametryzacją powierzchni sfery nie obejmującą punktu $(0,0,1)$. Co się 
dzieje w tym punkcie?
\end{itemize}
%To odwzorowanie nazywane jest \textit{projekcją stereograficzną}.
\end{exercise}


\begin{exercise}
Sprawdzić, że parametryzacja paraboloidy
\[x(u,v)=\left(u,v,u^2+v^2\right)\] jest regularna.
\end{exercise}

\begin{exercise}
Znaleźć parametryzację Monge'a stożka. 
\end{exercise}

\begin{exercise}
 Znaleźć parametryzację i wektor normalny do poniższych powierzchni:
 \begin{itemize}
  \item powierzchnia siodłowa
  \item Powierzchnia śrubowa 
  \item walec
  \item powierzchnia sfery o promieniu $R$.
 \end{itemize}
 Jak wyglądają w każdym przypadku linie parametr\'ow?
\end{exercise}

\subsection{Parametryzacja obrotowa}

\begin{exercise}
Wskazać reprezentację macierzową dla grupy $SO(3)$ (macierze obr\'ow o dowolny 
kąt wok\'oł każdej z osi).
\end{exercise}

\begin{exercise}
Korzystając z poprzedniego ćwiczenia wskazać parametryzację obrotową 
\begin{itemize}
 \item sfery o promieniu $R$
 \item hiperboloidy dwupowłokowej
 \item katenoidy (hiperboloida jednopowłokowa)
 \item paraboloidy
\item torusa
 \end{itemize}
Obliczyć wektor normalny i opisać linie parametru na tych powierzchniach.
\end{exercise}

\begin{exercise}
Zastanowić się, co się dzieje gdy tworzymy powierzchnię obrotową z 
krzywej kt\'ora 
\begin{itemize}
 \item posiada samoprzecięcia
 \item przecina oś obrotu.
\end{itemize}

\end{exercise}

\begin{exercise}
 Znaleźć og\'olny wz\'or na wektor normalny do powierzchni obrotowej.
\end{exercise}

\begin{exercise}
Wyznaczyć parametryzację powierzchni powstałej przez obrót krzywej 
\[\alpha(t)=(x,x+\sin{x},0)\] wokół prostej $l=(t,t,0)$.\\[0.1in]
\small{Wskazówka: wyobrazić sobie krzywą i powierzchnię w przestrzeni. Jakich 
obrotów wokół osi bazy standardowej trzeba dokonać, by uzyskać obrót wokół 
zadanej osi? Złożyć je.}
\end{exercise}

\begin{exercise}
 Napisać parametryzację obrotową stożka zawierającego wszystkie trzy osie 
wsp\'ołrzędnych.
\end{exercise}

% \begin{exercise}
% Znaleźć parametryzację powierzchni powstałej przez obrót krzywej 
% $(t,\sqrt{1+t^{2}},0)$ wokół drugiej osi współrzędnych. Wyznaczyć odwzorowanie 
% Gaussa dla tej parametryzacji. Wykazać, że jest ono różnowartościowe i 
% oszacować wielkość obrazu tego odwzorowania.
% \end{exercise}

\begin{exercise}
Pokazać, że macierze 
\begin{align*}
\mathcal{A}(\phi)&=\!\left[\!\!\!\!
\begin{array}{c c c}
\cos(\phi) &	\sin(\phi)&	0\\
-\sin(\phi) &	\cos(\phi)&	0\\
0 &			0 &	1
\end{array}\!\!\!\right], &
\mathcal{B}(\psi)&=\!\left[\!\!\!\!
\begin{array}{c c c}
1 &	0 		&	0\\
0 &	\cos(\psi) 	&	\sin(\psi)\\
0 &	-\sin(\psi) 	&	\cos(\psi)
\end{array}\!\!\!\right],
\end{align*}
\begin{align*}
\text{oraz }\quad\mathcal{C}(\chi)&=\!\left[\!\!\!\!
\begin{array}{c c c}
\cos(\chi) &	\sin(\chi)&	0\\
-\sin(\chi) &	\cos(\chi)&	0\\
0 &			0 &	1
\end{array}\!\!\!\right]
\end{align*}
(po odpowiedniej interpretacji) tworzą bazę $SO(3)$ (uwaga, tutaj nie ma błędu, 
to nie są znane nam obroty o zadany kąt wokół \textit{ustalonych} osi $x,y,z$!) 
tj. pokazać, że odpowiednio dobrany iloczyn $\mathcal{A},\mathcal{B}, 
\mathcal{C}$ zadaje dowolny obrót w przestrzeni $\mathbb{R}^3$.
(Obrót definiujemy jako odwzorowanie $\mathbb{R}^3\to \mathbb{R}^3$ zachowujące 
długość wektorów, kąty między nimi i ich wzajemną \textit{orientację}. Są to 
tzw. \textbf{Kąty Eulera}).
\end{exercise}

\subsection{Prostokreślność}

\begin{exercise}
 Wskazać parametryzację prostokreślną
 \begin{itemize}
  \item stożka
  \item walca
  \item powierzchni siodłowej
  \item katenoidy
  \item wstęgi M\"obiusa
 \end{itemize}

\end{exercise}

\begin{exercise}
Przedstaw jako powierzchnię prostokreślną powierzchnię daną równaniem: 

\begin{itemize}
 %\item $z^2=4x^2+y^2$
 \item $z=4x^2-y^2$
\end{itemize}
 \small{Wskazówka: Co to za powierzchnia? Znaleźć 
odpowiednie podstawienie i wyrazić prostokreślność w nowych zmiennych.}
\end{exercise}


\begin{exercise}
Korzystając z parametryzacji wstęgi M\"obiusa jako powierzchni prostokreślnej 
pokazać, że wektor normalny po obiegnięciu pełnego okręgu zmienił sw\'oj znak, a 
zatem jest to parametryzacja (!) \textit{nieorientowalna}.
\end{exercise}

\begin{exercise}
Podać parametryzację walca która nie będzie parametryzacją prostokreślną.
\end{exercise}

\subsection{Wektor normalny i płaszczyzna styczna}

\begin{exercise}
Wyznaczyć równanie płaszczyzny stycznej w punkcie 
\[\left(\frac{\sqrt{2}}{4},\frac{\sqrt{2}}{4},\frac{1}{2}\right)\] do sfery o 
środku $(0,0,0)$.\\[0.1in]
\small{Wskazówka: Jaki jest wektor normalny w tym punkcie? Następnie skorzystać 
z iloczynu skalarnego.}
 \end{exercise}

\begin{exercise}
 Wyznaczyć r\'ownanie płaszczyzny stycznej w punkcie $(1,1,1)$
 do powierzchni zadanej przez r\'ownanie $x^2+2y^2+z^2=3$.
\end{exercise}

\begin{exercise}
Znaleźć parametryzację powierzchni powstałej przez obrót krzywej 
$(t,\sqrt{1+t^{2}},0)$ wokół drugiej osi współrzędnych. Wyznaczyć odwzorowanie 
Gaussa dla tej parametryzacji. Wykazać, że jest ono różnowartościowe i 
oszacować wielkość obrazu tego odwzorowania.
\end{exercise}


\begin{exercise}
Wyznaczyć odwzorowanie Gaussa dla katenoidy 
\[x(u,v)=(u,\cosh{u}\cos{v},\cosh{u}\sin{v}).\]
Wykazać, że jest różnowartościowe. Oszacować wielkość obrazu tego odwzorowania 
dla $u>0$.
\end{exercise}

\subsection{Izometria i konforemność}

\begin{exercise}
 Obliczyć pierwszą formę podstawową dla następujących powierzchni
 \begin{itemize}
  \item sfera
 \item torus
\item powierzchnia śrubowa
\item katenodida
 \end{itemize}

 \end{exercise}
 
 \begin{exercise}
Niech będzie dana funkcja $f\colon \mathbb{R}^2 \to S^1 \times \mathbb{R}$ 
określona przez \[f(s,t)=(\cos s, \sin s, t).\]
\begin{itemize}
 \item Pokazać, że $f$ jest lokalną izometrią.
 \item Pokazać, że $f$ nie jest dyfeomorfizmem, więc $f$ nie może być izometrią.
\end{itemize}
 
\end{exercise}
 
\begin{exercise}
Niech $M\subset \mathbb{R}^3$ będzie powierzchnią gładką i niech $x\colon U\to 
M$ będzie lokalnym układem współrzędnych. Rozważmy krzywą gładką 
\[\alpha(t)=\left(\alpha_1(t),\alpha_2(t)\right)\subset U.\]

Pokazać, że długość krzywej $\overline{\alpha}=x\circ\alpha\colon \mathbb{R}\to 
M$ na powierzchni jest równa
\[L(\overline{\alpha})=\int_a^b\sqrt{I_{\alpha(t)}\left(\alpha_1 '(t),\alpha_2 
'(t)\right)}dt,\]
co można bezpośrednio zapisać:
\[\int_a^b\sqrt{\left(\alpha_1'\right)^2 g_{11}(\alpha(t))+ 2 
\alpha_1'\alpha_2'g_{12}(\alpha(t))+\left(\alpha_2'\right)^2 
g_{22}(\alpha(t))}\,dt.\]

Jaki jest związek między długościami krzywych na powierzchniach lokalnie 
izometrycznych?

\small {Podpowiedź: Zauważyć, że $(x\circ \alpha )(t)$ jest zwykłą krzywą w 
$\mathbb{R}^3$. Wektor prędkości jest r\'owny $x_1\alpha_1' +x_2\alpha_2'$. 
Znaleźć jego długość.}
\end{exercise} 
 

\begin{exercise}
 M\'owimy, że parametryzacja $x\colon \mathbb{R}^2\to M$ jest 
\textit{konforemna} (lub \textit{wiernokątna}) jeśli zachowuje kąty.

Zinterpretować tę geometryczną definicję w języku geometrii r\'ożniczkowej.
\end{exercise}
 
\begin{exercise}
 Pokazać, że parametryzacja jest konforemna wtedy i tylko wtedy, gdy $g_{11} 
=g_{22} $ oraz $g_{12} =g_{21} =0$.
\end{exercise}

\begin{exercise}
 Pokazać, że projekcja stereograficzna jest parametryzacją konforemną.
\end{exercise}

\subsubsection{Zastosowania kartograficzne}

\begin{exercise}(geograficzne -- projekcja Lamberta)
Rozważmy odwzorowanie kt\'ore punktowi na sferze wpisanej w walec 
przyporządkowuje odpowiadający punkt na tym walcu zachowując wsp\'ołrzędną $z$. 
Znajdź wz\'or opisujący to odwzorowanie.

\small{Uwaga -- odwzorowanie to nie obejmuje biegun\'ow.}
\end{exercise}


\begin{exercise}[geograficzne]
Aby policzyć pole powierzchni na sparametryzowanej powierzchni $x\colon U\to M$ 
można posłużyć się następującym wzorem:
\[A\left(x(u,v)\right)=\int_U \|x_u \times x_v \|\, dudv.\]

Pokaż, że projekcja Lamberta ze sfery wpisanej w walec na jego 
powierzchnię zachowuje pole, ale nie jest ani izometrią ani  
odwzorowaniem konforemnym.

\small{Najlepiej zacząć od wsp\'ołrzędnych sferycznych na sferze i 
parametryzacji $(\cos v,\sin v, \cos u)$ walca.}
\end{exercise}

\begin{exercise}[geograficzne -- projekcja Merkatora]
 Odwzorowanie Merkatora (wynalezione na długo przed początkami geometrii 
r\'ożniczkowej) było pierwszym odwzorowaniem, w kt\'orym linia prosta na mapie 
faktycznie była najkr\'otszą drogą na kuli ziemskiej.

Pokazać, że parametryzacja (lub odwzorowanie $R^2 \supset U \to S^2$)
\[x(u,v)=\left( \frac{\cos v}{\sinh u}, \frac{\sin v}{\sinh u}, 
\frac{\sinh u}{\cosh u}\right)\]
jest konforemna.

\small{Trudniejsze zadanie polega na wyprowadzeniu tej formuły. Niech 
$(\phi,\theta)$ będą wsp\'ołrzędnymi sferycznymi. Wtedy linie parametru $u$ 
muszą przejść na południki funkcją $f(u)$ -- korzystamy tutaj z symetrii sfery. 
Pokazać, że $f(u)=2\arctan(e^{-u})$.}

\end{exercise}


\subsection{Odwzorowanie Weingartena i Krzywizna Gaussa}
\begin{exercise}
Pokazać (z definicji), że odwzorowanie Weingartena $S_p$ jest odwzorowaniem 
liniowym.
\end{exercise}

\begin{exercise}
 Sprawdzić, że macierz odwzorowania Weingartena $S_p \colon T_p 
M \to T_p M$ wyraża się jako
\[S_p = I_p^{-1} II_p. \]
 
 \small{Wystarczy zapisać $S_p(x_u)=ax_u + bx_v$ i $S_p(x_v)=cx_u + d x_v$ i 
wyprowadzić układ r\'ownań liniowych na $a,b,c$ i $d$.}
\end{exercise}

\begin{exercise}
Korzystając z r\'ownań Weingartena obliczyć macierz odwzorowania Weingartena
$S_p$ dla następujących powierzchni
\begin{itemize}
 \item sfera
 \item powierzchnia śrubowa
 \item katenoida
 \item torus
\end{itemize}

\end{exercise}

\begin{exercise}
Korzystając z r\'ownania $S_p = I_p^{-1} II_p$ obliczyć formę macierzową 
odwzorowania Weingartena dla następujących powierzchni:
\begin{itemize}
 \item walec
 \item powierzchnia siodłowa
 \item powierzchnia śrubowa
%  \item torus
\end{itemize}

\end{exercise}

\begin{exercise}
Oblicz krzywiznę Gaussa i krzywiznę średnią powierzchni danej równaniem 
parametrycznym:
\begin{itemize}
 \item $x(u,v)=(v\cos u,v\sin u,\sin u)$,
 \item $x(u,v)=(u^2,2uv,2v^2)$
\end{itemize}
\end{exercise}

\begin{exercise}
Oblicz krzywizny główne i ich wektory w punkcie $p=(1,0,2)$ powierzchni danej 
równaniem
\begin{itemize}
 \item $z^2+2x^2+y^2=6,$
\item $z^2-2x^2-y^2=2$.
\end{itemize}
\small{Krzywizny główne to $k_1=k(u_1)=\max_{u}k(u)$, oraz 
$k_2=k(u_2)=\min_{u}k(u)$.\\$k(u)=S_p(u)\cdot u=\kappa_\alpha(0)\cos{\theta}$ 
to krzywizna normalna, w punkcie w kierunku wektora \textbf{jednostkowego} 
$u\in T_p$ z przestrzeni stycznej. $u_1$ i $u_2$ nazywamy wektorami krzywizn 
głównych. $\alpha$ jest krzywą na powierzchni spełniającą: $\alpha(0)=p$, oraz 
$\alpha'(0)=u$. $\theta$ jest kątem między $U(p)$, a $N_\alpha(0)$.\\[0.1in] 
Wskazówka: Powierzchnie te są poziomicami pewnej funkcji. Jaką postać może mieć 
wektor jednostkowy należący do przestrzeni stycznej w punkcie $p$?} 
\end{exercise}

\begin{exercise}
Proszę wybrać punkt na powierzchni danej równaniem\[z^2-5x^2+y^2=5,\]a 
następnie policzyć w nim krzywizny główne i ich wektory.
\end{exercise}


\begin{exercise}
Dla powierzchni zadanej równaniem
\[2 x^2 - y^2 - z^2 = 2\]
W punkcie $(2,0,2)$ policzyć krzywiznę Gaussa i średnią a także znaleźć 
krzywizny główne i ich wektory.
\end{exercise}


\begin{exercise}
 Pokazać że powierzchnia prostokreślna ma krzywiznę Gaussa $\leqslant 0$.
\end{exercise}


\begin{exercise}
 Sprawdzić, że powierzchnia pseudosfery (tj. powierzchnia otrzymana przez 
obr\'ot traktrysy -- krzywej pościgu) zadana wzorem
\[x(u,v)=\left(u-\tanh u, \frac{\cos v}{\cosh u},\frac{\sin v}{\cosh u}\right)\]
ma stałą krzywiznę Gaussa r\'owną $-1$.
\end{exercise}


\subsection{Theorema Egregium Gaussa}

\begin{exercise}
 Udowodnić R\'ownanie Gaussa
 \begin{equation*}
l_{11}l_{22}-l_{12}^2=
\sum_{r=1}^{2}g_{1r}
\left[
\frac{\partial \Gamma^r_{22}} {\partial u_{1}}- \frac{\partial\Gamma 
^r_{21}}{\partial u_{2}}+
\sum_{m=1}^2 
\left(\Gamma^m_{22}\Gamma^r_{m1}-\Gamma^m_{21}\Gamma^r_{m2}\right)\right].
\end{equation*}
\footnotesize{Podpowiedź: należy porównać współczynniki stojące przy $x_1$ i 
$x_2$ w rozwinięciach $x_{ijk} $ i $x_{ikj} $ względem bazy 
$\{x_1,x_2,n\}$. Następnie podstawić $(i=2,j=1,k=2)$.}

\end{exercise}

\begin{exercise}
 Prześledzić dow\'od Theorema Egregium Gaussa i wyprowadzić jawny wz\'or na 
krzywiznę korzystający tylko ze wsp\'ołczynnik\'ow metrycznych.
\end{exercise}

\begin{exercise}
Niech 
\begin{align*}
M=&\{y(u,v)=(u \sin v, u\cos v,\ln u)\colon u\in \mathbb{R}_+, v\in 
(-\pi,\pi)\},\\
N=&\{x(u,v)=(v \sin u, v\cos u, u)\colon u\in \mathbb{R}_+, v\in (-\pi,\pi)\},
\end{align*}
oraz zdefiniujmy funkcję $f\colon M\to N$ jako \[f(y(u,v))= x(v,u).\]
Sprawdzić, że 
\[K\left( f(y(u,v))\right)=K(x(v,u))=\frac{-1}{(1+u^2)^2}=K(y(u,v)),\]
a mimo to $f$ nie jest (lokalną) izometrią.
\end{exercise}

\subsection{Geodezyjne}

\begin{exercise}
 Korzystając z r\'ownań geodezyjnych pokazać, że proste i tylko proste są 
geodezyjnymi na płaszczyźnie.
\end{exercise}

\begin{exercise}
 Sprawdzić jak wyglądają r\'ownania geodezyjnych dla sfery o promieniu $1$. 
Spr\'obować wyprowadzić, że krzywe je spełniające są okręgami wielkimi.
\end{exercise}



\begin{exercise}
Niech $\gamma$ będzie (niestałą) krzywą geodezyjną na powierzchni $M$. 
\begin{itemize}
 \item Pokazać, że $\gamma$ ma stałą prędkość.
 \item Reparametryzacja tej krzywej $\gamma\circ h (t)$ jest geodezyjną wtedy i 
tylko wtedy, gdy $h$ jest funkcją afiniczną. 
\end{itemize}

\end{exercise}

\begin{exercise}
Niech $M\subset \mathbb{R}^3$ będzie powierzchnią obrotową z lokalnym układem 
współrzędnych 
\[x(t,\vartheta)=\big(\alpha_1(t),\alpha_2(t)\cos \vartheta ,
\alpha_2(t)\sin\vartheta \big),\]
oraz załóżmy, że krzywa profilu $x(t,0)=(\alpha_1(t),\alpha_2(t),0)$ jest 
unormowana. Pokazać, że r\'ownania geodezyjnych mają dla $M$ postać
\begin{align*}
g''_1-\alpha_1\alpha_1'\left(g'_2\right)^2&=0\\
g''_2+2\frac{\alpha_1'}{\alpha_1}g'_1g'_2&=0.
\end{align*}
\end{exercise}

\begin{exercise}
Korzystając z powyższej formy r\'ownań geodezyjnych pokazać, że każda krzywa 
profilu (południk) na powierzchni $M$ może być sparametryzowana jako krzywa 
geodezyjna.
\end{exercise}

\begin{exercise}
Sprawdzić, że krzywa na powierzchni obrotowej zadana przez r\'ownanie 
$t=t_0$ (tzn. linia parametru) jest geodezyjną wtedy i tylko wtedy, gdy 
$\alpha'_1(t_0)=0$. Co ten wynik oznacza geometrycznie?

\end{exercise}

\begin{exercise}
 Zastanowić się jak mogą wyglądać geodezyjne na torusie. Spr\'obować 
wywnioskować pewne własności geodezyjnych na torusie z postaci r\'ownań 
geodezyjnych.
\end{exercise}


\subsection{Twierdzenie Gaussa-Bonneta}

\begin{exercise}
Znaleźć triangulację sfery o minimalnej liczbie ścian. Czy można ją zrealizować 
przez tr\'ojkąty geodezyjne?
\end{exercise}

\begin{exercise}
Znaleźć triangulację (dowolną, niekoniecznie geodezyjną) torusa.
\end{exercise}

\begin{exercise}
Znaleźć przedstawienie geometryczne torusa jako powierzchni 
powstałej przez utożsamienie bok\'ow $n$\nobreakdash-kąta foremnego. Dla jakich 
$n$ to się uda?
\end{exercise}

\begin{exercise}
 Policzyć charakterystykę Eulera dla sfery i torusa.
\end{exercise}

\begin{exercise}
 Pokazać geometrycznie jak skonstruować $2$\nobreakdash-torus (poprzez sumę 
sp\'ojną). Zinterpretować to w języku utożsamiania bok\'ow wielokąt\'ow.
\end{exercise}

\begin{exercise}
Korzystając z prezentacji $n$\nobreakdash-torusa znaleźć jego triangulację i 
policzyć charakterystykę Eulera. 
\end{exercise}

\begin{exercise}
Jakie wnioski odnośnie krzywizny $n$\nobreakdash-torus\'ow\footnote{W 
rzeczywistości $n$\nobreakdash-torusy to jedyne zamknięte powierzchnie 
orientowalne, więc jest to dosyć og\'olny wniosek.} można wyciągnąć na 
podstawie Twierdzenia Gaussa-Bonneta?
\end{exercise}

\subsection{Geometria hiperboliczna i płaskie zanurzenia}

\begin{exercise}
 Policzyć $I\!I$ formę podstawową, symbole Christoffela i krzywiznę Gaussa dla 
płaszczyzny hiperbolicznej zdefiniowanej jako półpłaszczyznę 
\[\mathcal{H}=\left\{(u,v)\in \mathbb{R}^2\colon v>0 \right\},\] wyposażoną w 
pierwszą formę podstawową
\[I_{\mathcal{H}}=\left\{I(u,v)\right\}_{(u,v)\in 
\mathcal{H}}=\left\{\left[
\begin{array}{cc}
\frac{1}{v^2}&0\\
0 & \frac{1}{v^2}
\end{array}
\right]\right\}
\]

\end{exercise}

\begin{exercise}
 Korzystając z poprzedniego zadania wyznaczyć r\'ownania geodezyjnych dla 
$\mathcal{H}$. Następnie sprawdzić, że prosta w parametryzacji 
\[v(t)=\left(C,C_1 e^{C_2 t}\right)\] jest geodezyjną, gdzie $C, C_{1} , C_2$ 
są stałymi.
\end{exercise}


\begin{exercise}[Transformacje M\"obiusa]
Pokazać, że specjalna transformacja M\"obiusa zdefiniowana jako funkcja
$T^{a,b}_{c,d}\colon \mathbb{C}\to\mathbb{C}$ i zadana wzorem
\[T^{a,b}_{c,d}(z)= \frac{az+b}{cz+d}\]
zachowuje g\'orną p\'ołpłaszczyznę, tj.
\[T^{a,b}_{c,d}\Big|_{\mathcal{H}}\colon \mathcal{H}\to \mathcal{H}\]
jest dobrze zdefiniowanym odwzorowaniem.
\end{exercise}

\begin{exercise}
Pokazać, że złożenie transformacji M\"obiusa odpowiada mnożeniu macierzy.
\end{exercise}


\begin{exercise}
Pokazać, że specjalne transformacje M\"obiusa przenoszą geodezyjne na 
geodezyjne. 

\small{\textbf{Podpowiedź:} Pokazać, że następujące macierze 
generują grupę $PSL(2,\mathbb{R})$: 
\[\left\{\left(\begin{array}{cc}
     1 & c\\
	0 & 1\\
     \end{array}
\right),\quad \left(\begin{array}{cc}
     \lambda & 0\\
	0 & ^1\!\!\big/_{\!\!\lambda}\\
     \end{array}
\right),\quad\left(\begin{array}{cc}
     0& -1\\
	1 & 0\\
     \end{array}
\right)\right\}.\]
Następnie zinterpretować geometrycznie działanie poszczególnych macierzy.}
\end{exercise}

\begin{exercise}[Model Kleina]
Koło otwarte $\mathcal{K}=\{(x,z)\colon x^2 + z ^2 < 1\}$, wraz z pierwszą 
formą podstawową
\[I_{\mathcal{K}}=I_{(x,z)}
=\left[
\begin{array}{cc}
\frac{4}{(1-x^2-z^2)^2} & 0\\
0 & \frac{4}{(1-x^2-z^2)^2}
\end{array}
\right]
\]
nazywamy \textit{Modelem Kleina} geometrii hiperbolicznej.

Pokazać, że odwzorowanie $W\colon \mathcal{H}\to \mathcal{K}$ określone wzorem
\[W(x+iy)=\frac{x+i(y-1)}{x+i(y+1)}\]jest izometrią.

\small{Podpowiedź: $W$ jest transformacja M\"obiusa stowarzyszona z 
macierzą 
$\left[
\begin{array}{cc}
1 &-1\\
1 & 1
\end{array}
\right].$}

\end{exercise}

\begin{exercise}
Dlaczego przy konstrukcji \textit{gładkiego} $n$\nobreakdash-torusa poprzez 
identyfikację bok\'ow wielokąta foremnego wymagamy, by suma kąt\'ow 
wewnętrznych była r\'owna $2\pi$?? Co się dzieje kiedy suma jest mniejsza? Co 
jeśli większa?
\end{exercise}



\subsection{Zadania r\'ożne}
\begin{exercise}
Niech będą dane dwa współosiowe, równoległe okręgi w $\mathbb{R}^3$ odległe od siebie o $A$, o promieniach odpowiednio $r$ i $R$. Podać parametryzację katenoidy zawierającej je oba. Czy zawsze da się to zrobić?
\end{exercise}

% \begin{exercise}
% Wybierając pewną powierzchnię nieorientowalną, faktycznie udowodnić jej nieorientowalność (To zadanie jest już zdecydowanie obowiązkowe dla osób z projektem \textit{Powierzchnie nieorientowalne}).
% \end{exercise}

\begin{exercise}
Zidentyfikować wszystkie powierzchnie, dla których linie normalne (proste 
wyznaczone przez wektory normalne przesunięte do odpowiedniego punktu na 
powierzchni) przechodzą przez jeden ustalony punkt $P$.
\end{exercise}

\begin{exercise}
Policzyć odwzorowanie Weingartena dla dowolnej powierzchni powstałej przez obrót krzywej \[\alpha(t)=(f(t),g(t),0)\] wokół osi $OX$. Jak wyglądają krzywizny główne i ich wektory? (dla ułatwienia można założyć, że $\alpha$ jest krzywą unormowaną).
\end{exercise}

\begin{exercise}
Policzyć odwzorowanie Weingartena dla dowolnej powierzchni prostokreślnej. Jak wyglądają krzywizny główne i ich wektory?
\end{exercise}


\begin{exercise}
Udowodnić bądź znaleźć kontrprzykład:
\begin{quote}
Jeśli $M$ jest powierzchnią o dodatniej krzywiźnie Gaussa ($K(p)>0$ dla 
wszystkich $p\in M$), wtedy każda krzywa leżąca na powierzchni ma również 
krzywiznę dodatnią.
\end{quote}

\end{exercise}



\begin{exercise}
Niech $M$ będzie dowolną powierzchnią prostokreślną. Udowodnić lub wskazać 
kontrprzykład: 
\begin{quote}
Odwzorowanie Weingartena wzdłuż prostych będących liniami parametru jest 
zerowe. 
\end{quote}

\end{exercise}


\begin{exercise}
Pokazać, że jeśli odwzorowanie Weingartena powierzchni $M$ jest postaci $\frac{-1	}{R}[\text{id}]$, gdzie $[\text{id}]$ oznacza macierz identycznościową, to $M$ jest powierzchnią sfery (lub jej fragmentem).
\end{exercise}

\begin{exercise}
 Pokazać, że zamknięta powierzchnia $M$ (tzn. zwarta i bez brzegu) zanurzona 
(gładko!) w $\mathbb{R}^3$ posiada taki punkt $p\in M$, że $K(p)> 0$.
\end{exercise}


\begin{exercise}[Wiązki okręg\'ow nad krzywą zamkniętą]
Wstęga M\"obiusa powstawała jako powierzchnia prostokreślna przez obiegnięcie 
okręgu r\'ownocześnie przekręcając proste ,,normalne" o połowę kąta pełnego.

Rozważmy następujuącą powierzchnię ,,okręgo-kreślną``, czyli składającą się z 
odpowiednio przekręconych okręgów nad (bazowym) okręgiem\footnote{
Prawidłowa nazwa brzmi: $S^1$-wiązka nad okręgiem.}. Niech \[\alpha\colon 
[0,2\pi)\to \mathbb{R}^3\] będzie krzywą zamkniętą bez samoprzecięć o zerowej 
torsji. 

\begin{itemize}
 \item Dla każdego $t\in [0,2\pi)$ do krzywej $\alpha$ 
dołączamy okrąg o promieniu $\varepsilon$ (odpowiednio małym) przechodzący 
przez punkt $\alpha(t)$, leżący w płaszczyźnie $N_{\alpha} (t)\times B_{\alpha} 
(t)$ (płaszczyzna normalna). Jak wygląda powierzchnia utworzona z tych 
okręg\'ow? Przyjąć, że $\alpha$ jest okręgiem jednostkowym i znaleźć 
parametryzację utworzonej powierzchni.
\item Zał\'ożmy teraz, że zamiast okręgu umieszczamy w każdym punkcie $\alpha$ 
odpowiednio małą lemniskatę Bernoulliego (kt\'orej środek leży punkcie 
$\alpha(t)$). Jak wygląda utworzona powierzchnia w zależności od sposobu 
położenia lemniskaty? jak wyglądają punkty samoprzecięcia?

\item Przyjmijmy, że $\alpha(t)$ jest okręgiem jednostkowym i skorzystajmy z 
symetrii lemniskaty aby wzdłuż krzywej $\alpha(t)$ obr\'ocić ją o połowę kąta 
pełnego. Znaleźć parametryzację utworzonej powierzchni. Jest to tzw. Butelka 
Kleina.

\item Korzystając z powyższej parametryzacji pokazać, że butelka Kleina jest 
sklejeniem dw\'och wstęg M\"obiusa wzdłuż ich brzegu.
\end{itemize}



\small {Uwaga: Butelki Kleina nie uda się zanurzyć w $\mathbb{R}^3$ bez 
samoprzecięć.}
\end{exercise}


\end{document}

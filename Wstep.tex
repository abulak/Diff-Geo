\thispagestyle{empty}\lowsection{Uwagi wstępne}
Poniższy skrypt jest przygotowany do wykładu \textit{Wstęp 
do Topologii i Geometrii Różniczkowej} kt\'ory odbywa się na  Wydziale 
Matematyki i Informatyki Uniwersytetu im. Adama Mickiewicza w Poznaniu.

Korzystanie z notatek i dołączonych w oddzielnym pliku zadań w celach 
poszerzania swojej wiedzy jest dozwolone a nawet rekomendowane. Osoby chcące 
użyć tych materiałów w celach dydaktycznych proszone są o wcześniejszy kontakt 
ze mną, a przynajmniej poinformowanie mnie o tym fakcie.

\thispagestyle{empty}\lowsection{Struktura wykładu}

Materiał podzielony jest na 15 wykładów, które "pęcznieją" wraz ze swoim 
numerem, tj. ich długość i objętość zawartej treści jest mniej więcej 
proporcjonalna do ich numeru porządkowego. Dlatego zalecane jest przyśpieszanie 
pierwszych wykładów. 

Pierwsze cztery wykłady dotyczą geometrii krzywych w przestrzeni $\R^3$ i 
kończą się na dowodzie twierdzenia klasyfikacyjnego. Wprowadzone są Tr\'ojn\'og 
Freneta i klasyczne niezmienniki krzywych -- krzywizna i torsja. Dow\'od 
Twierdzenia Klasyfikacyjnego zamieszczony w tych notatkach wymaga nieco wprawy w 
operowaniu r\'ownaniami r\'ożniczkowymi, więc może być przy pierwszym czytaniu 
(czy na wykładzie) pominięty.

Następne sześć rozdział\'ow poświęconych jest lokalnej teorii powierzchni 
gładkich w $\R^3$. Omawiane są: lokalne układy wsp\'ołrzędnych, wektory styczne 
i normalne, I i II forma podstawowa, odwzorowanie Weingartena i wreszcie 
krzywizna Gaussa powierzchni. Część ta kończy się Theorema Egregium Gaussa i 
Twierdzeniem Klasyfikującym powierzchni (bez dowodu). Jeden cały wykład 
poświęcony jest ideom, kt\'ore stały za definicją krzywizny Gaussa, oraz 
(kwazi-formalnej) pr\'obie dowodu, że r\'ożniczkowa definicja faktycznie mierzy 
podobną wielkość. Wydaje mi się, że warto te związki podczas wykładu uwypuklać.

Nie m\'owimy niemal słowa o rozmaitościach ale staramy się w cieniu przemycić 
nieco pojęć topologii r\'ożniczkowej takich jak np. poziom (lub wartość) 
krytyczny. Używany język z teorii rozmaitości (jak np. funkcje przejścia i 
definicja gładkości odwzorowania między powierzchniami) jest wprowadzony by 
nadać nieco rygoru naszym definicjom. Jeśli czytelnik nie czuje się w nim pewnie 
może pominąć te fragmenty wykładu -- ważne, by zapamiętał wnioski napisane 
potocznym językiem.

W trakcie wykładu pojawiają się r\'ownież dwie powt\'orki z algebry liniowej -- 
ilustrowane przykładami mają na celu wprowadzić pewne minimum pojęciowe, by 
można było kontynuować wykład bez zbytniej ekwilibrystyki. Pożądanym byłoby, 
gdyby czytelnik te części faktycznie m\'ogł traktować jako powt\'orki z 
wykład\'ow z algebry liniowej.



Następne trzy wykłady dotyczą geodezyjnych i Twierdzenia Gaussa-Bonneta. 
Starałem się umotywować geometrycznie definicję geodezyjnych i na przykładach 
je zilustrować. R\'ownież i w tej części korzysta się z wynik\'ow 
dotyczących r\'ownań r\'ożniczkowych, ale ogranicza się to do bezpośredniego 
rozwiązywania r\'ownań, kt\'ore dla czytelnika po trzech kursach analizy 
matematycznej nie powinny stanowić problemu.

Wykłady 13 (o twierdzeniu Gaussa-Bonneta), 14 i 15 (o geometrii hiperbolicznej 
i powierzchniach o stałej krzywiźnie) są raczej poglądowe, dowodów tam jest 
niewiele. Mają one raczej charakter informacyjny i motywujący do dalszej nauki 
geometrii różniczkowej. Chciałem, by raczej pokazywały wpływ lokalnej geometrii 
na globalny kształt przestrzeni.

\thispagestyle{empty}\lowsection{Skład}

Zdaję sobie sprawę, że typografia notatek/skryptu pozostawia wiele do życzenia. 
Jest to skutek tego, że i skrypt i slajdy powstają z tych samych plików 
źródłowych. Coś, co wygląda dobrze na slajdzie niekoniecznie musi dobrze 
wyglądać w tekście i vice-versa. Wraz z kolejnymi latami mam jednak 
nadzieję ten aspekt polepszać. Wszystkie uwagi na ten temat są jak najbardziej 
mile widziane.


\vfill

\thispagestyle{empty}\lowsection{Prawa autorskie}
Prezentowany w tej książce materiał jest na tyle klasyczny, że postanowiłem nie 
umieszczać cytowań w tekście. W zasadzie każdy wsp\'ołczesny podręcznik do 
Geometrii r\'ożniczkowej w wymiarze $3$ zawiera znaczną część materiału 
zawartego w tych notatkach. Osoby wnikliwie por\'ownujące dostępne 
teksty dostrzegą podobieństwo prezentacji materiału tego wykładu do dw\'och 
książek:

\begin{itemize}
 \item Ethan D. Bloch, \textit{A First Course on Geometric Topology and 
Differential Geometry}, Birkh\"auser 1997
\item Theodore Shifrin, \textit{\textsc{Differential Geometry}: A First Course 
in Curves and Surfaces}, Preliminary Version 2008.
\end{itemize}

Zwłaszcza z tej drugiej pozycji zaczerpnąłem część zadań jak i motywację do 
tworzenia własnych.\thispagestyle{empty}
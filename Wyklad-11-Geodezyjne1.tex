\mode*
\mode<all>{\topsection{Geodezyjne I}}
\mode<all>{\midsection{Idea}}
\begin{frame}[<+->]
Geodezyjne na powierzchniach to pewne analogi linii prostych na płaszczyźnie. Linie proste mają trzy ważne własności:
\begin{itemize}
\item [\textbf{(1)}]każde dwa punkty łączy dokładnie jedna prosta;
\item [\textbf{(2)}]odcinek tej prostej zawarty między punktami jest najkrótszą drogą między nimi;
\item [\textbf{(3)}]kiedy podróżujemy wzdłuż prostej ``nie skręcamy w prawo bądź w lewo''.
\end{itemize}
\end{frame}
%%%%%%next-slide%%%%%
\begin{frame}

Chociaż proste na płaszczyśnie spełniają równocześnie te trzy warunki, dla ogólnej powierzchni może nie być to możliwe. Rozważmy przykład $S^2$:
\begin{itemize}
\pause\item Najkrótszą drogą łączącą dwa punkty o tej samej ``długości geograficznej'' jest droga wzdłuż tego południka, czyli tzw. okręgu wielkiego. Z symetrii sfery wynika, że każde dwa punkty na sferze leżą na pewnym okręgu wielkim.
\pause\item Dwa punkty dzielą okrąg wielki na dwie (najczęściej nierówne) części, więc nie każdy fragment okręgu jest najkrótszą drogą.
\pause\item Między biegunem północnym i południowym istnieje nieskończenie wiele południków równej długości, więc każdy z nich może być uważany za najkrótszą drogę między biegunami.
\end{itemize}

\end{frame}
%%%%%%next-slide%%%%%
Jak widzimy nawet przy najprostszej powierzchni jaką jest sfera, konfrontacja intuicji ``najkrótszej drogi'' z własnościami okręgów wielkich sprawia, że nie możemy zakładać równoczesnego spełnienia wszystkich trzech własności.  

\begin{frame}[<+->]
Chociaż najmniej intuicyjną, wybierzemy własność \textbf{(3)} jako definicję. Oto najważniejszy powód naszej decyzji: 
\begin{quote}
``Nie krzywienie się'' jest własnością \textit{lokalną}, natomiast własności jedyności \textbf{(1)} i najkrótszej drogi \textbf{(2)} są \textit{globalne}, i odwołują się do kształtu całej powierzchni.
\end{quote}


\end{frame}
Z naszego punktu widzenia na powierzchnię (poprzez lokalne układy współrzędnych), własności lokalne, które można sprawdzać w otoczeniu każdego punktu z osobna są znaczenie bardziej przystępne (obliczeniowo), niż te, które wymagają brania pod uwagę związków pomiędzy punktami (być może) bardzo dalekimi.
%%%%%%next-slide%%%%%

\mode<all>{\midsection{Pochodna kowariantna}}

\begin{frame}[<+->]
Ale jak uogólnić (czy też doprecyzować) własność \textbf{(3)} dla dowolnej powierzchni? Wykorzystamy ideę równoległego transportu wektora wzdłuż krzywej. \pause Postrzeganie równoległości tak na prawdę zależy od położenia obserwatora. 
\begin{center}

\definecolor{c0000ff}{RGB}{0,0,255}
\definecolor{cff0000}{RGB}{255,0,0}
\usetikzlibrary{arrows}

\mode<presentation>{\begin{tikzpicture}[y=0.80pt, x=0.8pt,yscale=-1,scale=0.4, inner sep=0pt, outer sep=0pt]}
\mode<article>{\begin{tikzpicture}[y=0.80pt, x=0.8pt,yscale=-1,scale=0.5, inner sep=0pt, outer sep=0pt]}
\begin{scope}[shift={(-127.26916,-608.18524)}]% layer1
  \begin{scope}% g930
    \begin{scope}[shift={(-630.60299,773.9938)},opacity=0.500,transparency group]% g1453-3
      % path53-6
      \path[fill=black] (814.4212,58.5435) .. controls (820.3754,93.7311) and
        (836.8215,127.0965) .. (861.8065,152.5185) .. controls (861.8065,152.5185) and
        (861.8065,152.5185) .. (861.8065,152.5185) .. controls (877.3745,168.3508) and
        (896.0728,181.0812) .. (916.5208,189.7208) .. controls (916.5208,189.7208) and
        (916.5208,189.7208) .. (916.5208,189.7208) .. controls (938.2006,198.8777) and
        (961.7638,203.3783) .. (985.2680,203.2928) .. controls (985.2680,203.2928) and
        (985.2680,203.2928) .. (985.2680,203.2928) .. controls (988.4665,203.2814) and
        (991.6647,203.1752) .. (994.8567,202.9746) .. controls (1015.0320,201.7064)
        and (1034.9702,196.7427) .. (1053.2767,188.1710) .. controls
        (1073.3535,178.7669) and (1091.3975,165.1492) .. (1106.2074,148.6496) ..
        controls (1106.2074,148.6496) and (1106.2074,148.6496) .. (1106.2074,148.6496)
        .. controls (1120.8905,132.2919) and (1132.4412,113.1366) ..
        (1140.1348,92.5338) .. controls (1147.1845,73.6587) and (1150.9849,53.5407) ..
        (1151.0552,33.3886) .. controls (1151.0652,31.5089) and (1151.0372,29.6289) ..
        (1150.9802,27.7496) .. controls (1150.9802,27.7496) and (1150.9802,27.7496) ..
        (1150.9802,27.7496) .. controls (1150.3459,6.7970) and (1145.7929,-14.0185) ..
        (1137.7201,-33.3655) .. controls (1129.6473,-52.7126) and (1118.0511,-70.5899)
        .. (1103.3264,-85.6015) .. controls (1103.3264,-85.6015) and
        (1103.3263,-85.6015) .. (1103.3263,-85.6015) .. controls (1088.9567,-100.2437)
        and (1071.7700,-112.1497) .. (1052.9265,-120.3583) .. controls
        (1052.9265,-120.3583) and (1052.9264,-120.3583) .. (1052.9264,-120.3583) ..
        controls (1034.2127,-128.5074) and (1013.9178,-132.9553) ..
        (993.5271,-133.5322) .. controls (992.5638,-133.5595) and (991.6002,-133.5783)
        .. (990.6366,-133.5886) .. controls (968.3513,-133.8241) and
        (946.1243,-129.4717) .. (925.4123,-121.4288) .. controls (904.6002,-113.3501)
        and (885.2493,-101.5604) .. (868.6019,-86.7416) .. controls
        (868.6019,-86.7416) and (868.6019,-86.7416) .. (868.6019,-86.7416) .. controls
        (852.0572,-72.0170) and (838.1847,-54.2239) .. (828.5521,-34.3115) .. controls
        (828.5521,-34.3115) and (828.5521,-34.3115) .. (828.5521,-34.3115) .. controls
        (821.1762,-19.0603) and (816.3964,-2.5704) .. (814.5363,14.2211) .. controls
        (813.5712,19.9703) and (813.4850,24.2752) .. (813.6871,27.0631) .. controls
        (813.8891,29.8509) and (814.3785,31.1348) .. (814.8395,30.9987) .. controls
        (815.3006,30.8625) and (815.7362,29.3143) .. (816.0857,26.4386) .. controls
        (816.4352,23.5630) and (816.7007,19.3626) .. (817.0351,13.8445) .. controls
        (818.9464,-2.4347) and (823.6513,-18.4133) .. (830.8336,-33.1926) .. controls
        (830.8336,-33.1926) and (830.8336,-33.1926) .. (830.8336,-33.1926) .. controls
        (840.3409,-52.7626) and (854.0447,-70.2570) .. (870.3712,-84.7362) .. controls
        (870.3712,-84.7362) and (870.3712,-84.7362) .. (870.3712,-84.7362) .. controls
        (886.8048,-99.3078) and (905.9029,-110.8894) .. (926.4191,-118.8074) ..
        controls (946.8363,-126.6852) and (968.7161,-130.9288) .. (990.5966,-130.6503)
        .. controls (991.0043,-130.6453) and (991.4120,-130.6384) ..
        (991.8197,-130.6302) .. controls (1012.3883,-130.2142) and
        (1032.8870,-125.7854) .. (1051.7088,-117.5386) .. controls
        (1051.7088,-117.5386) and (1051.7088,-117.5386) .. (1051.7088,-117.5386) ..
        controls (1070.1704,-109.4514) and (1087.0055,-97.7342) ..
        (1101.0624,-83.3551) .. controls (1115.4667,-68.6278) and (1126.8038,-51.0718)
        .. (1134.6899,-32.0835) .. controls (1142.5760,-13.0952) and
        (1147.0149,7.3240) .. (1147.6166,27.8466) .. controls (1147.6166,27.8466) and
        (1147.6166,27.8466) .. (1147.6166,27.8466) .. controls (1147.6326,28.3811) and
        (1147.6456,28.9157) .. (1147.6556,29.4504) .. controls (1148.0610,50.0902) and
        (1144.4865,71.3087) .. (1137.0911,91.3998) .. controls (1129.6869,111.5099)
        and (1118.4636,130.5046) .. (1104.0814,146.7470) .. controls
        (1104.0814,146.7470) and (1104.0814,146.7470) .. (1104.0814,146.7470) ..
        controls (1089.5700,163.1345) and (1071.8025,176.6809) .. (1052.2415,185.9599)
        .. controls (1033.5864,194.8132) and (1013.3474,199.6927) ..
        (993.3781,200.7747) .. controls (990.6689,200.9215) and (987.9644,200.9976) ..
        (985.2676,201.0040) .. controls (961.5766,201.0586) and (938.4845,196.3434) ..
        (917.5424,187.3157) .. controls (897.3796,178.6265) and (879.2041,165.9873) ..
        (864.0382,150.3361) .. controls (853.1146,139.0685) and (843.8358,126.2066) ..
        (836.3499,112.1612) .. controls (828.8641,98.1157) and (823.1671,82.8815) ..
        (819.5655,66.8410) .. controls (818.7252,63.1879) and (817.7457,58.5408) ..
        (816.9211,53.7921) .. controls (816.0965,49.0434) and (815.4294,44.1954) ..
        (814.9282,40.1943) .. controls (814.4270,36.1933) and (814.0732,33.0404) ..
        (813.7050,31.7033) .. controls (813.5209,31.0348) and (813.3320,30.8205) ..
        (813.1355,31.1839) .. controls (812.9389,31.5473) and (812.7330,32.4884) ..
        (812.5534,34.1326) .. controls (812.4051,37.7311) and (812.5169,41.8343) ..
        (812.8565,46.0352) .. controls (813.1964,50.2361) and (813.7641,54.5330) ..
        (814.4212,58.5435) -- cycle;

      % path55-23
      \path[fill=black] (1018.1258,197.5435) .. controls (1032.1810,189.3284) and
        (1044.9190,178.8829) .. (1055.1265,166.2307) .. controls (1066.2029,152.4809)
        and (1074.1578,136.3667) .. (1078.9331,119.4360) .. controls
        (1078.9331,119.4360) and (1078.9331,119.4360) .. (1078.9331,119.4360) ..
        controls (1079.3917,117.8099) and (1079.8224,116.1770) .. (1080.2257,114.5379)
        .. controls (1083.8144,99.9537) and (1085.4669,84.9341) .. (1085.5968,69.9374)
        .. controls (1085.7286,54.9173) and (1084.3602,39.8820) .. (1081.5801,25.1175)
        .. controls (1081.5801,25.1175) and (1081.5801,25.1175) .. (1081.5801,25.1175)
        .. controls (1080.5850,19.8377) and (1079.3114,14.6203) .. (1077.7951,9.4756)
        .. controls (1074.5536,-1.5229) and (1070.4034,-12.2558) ..
        (1065.6350,-22.6709) .. controls (1065.6350,-22.6709) and (1065.6350,-22.6709)
        .. (1065.6350,-22.6709) .. controls (1057.3777,-40.7226) and
        (1046.9982,-57.9471) .. (1033.9537,-73.0247) .. controls (1031.2633,-76.1370)
        and (1028.4516,-79.1565) .. (1025.5200,-82.0570) .. controls
        (1013.8830,-93.5707) and (1000.5506,-103.4031) .. (985.6962,-110.2445) ..
        controls (985.6962,-110.2445) and (985.6962,-110.2445) .. (985.6962,-110.2445)
        .. controls (976.2165,-114.6053) and (966.1590,-117.6954) ..
        (955.8750,-119.2930) .. controls (947.1619,-120.6449) and (938.3137,-120.9197)
        .. (929.5634,-120.2180) .. controls (927.1037,-120.3482) and
        (925.3145,-120.0704) .. (924.1845,-119.6647) .. controls (923.0544,-119.2591)
        and (922.5787,-118.7250) .. (922.6920,-118.2745) .. controls
        (922.8053,-117.8240) and (923.5041,-117.4560) .. (924.7323,-117.3308) ..
        controls (925.9605,-117.2057) and (927.7151,-117.3228) .. (929.9989,-117.8028)
        .. controls (938.4820,-118.4245) and (947.0413,-118.1094) ..
        (955.4496,-116.7634) .. controls (965.4833,-115.1594) and (975.2953,-112.0924)
        .. (984.5377,-107.7900) .. controls (984.5377,-107.7900) and
        (984.5377,-107.7900) .. (984.5377,-107.7900) .. controls (998.8274,-101.1442)
        and (1011.6802,-91.6126) .. (1022.9499,-80.4534) .. controls
        (1025.9904,-77.4427) and (1028.9018,-74.2995) .. (1031.6828,-71.0549) ..
        controls (1044.4532,-56.1489) and (1054.5918,-39.1165) .. (1062.6757,-21.2767)
        .. controls (1062.6757,-21.2767) and (1062.6757,-21.2767) ..
        (1062.6757,-21.2767) .. controls (1067.1138,-11.4699) and (1071.0008,-1.4168)
        .. (1074.0909,8.8347) .. controls (1075.7460,14.3254) and (1077.1804,19.9663)
        .. (1078.3065,25.7332) .. controls (1081.1101,40.0716) and (1082.7695,55.0219)
        .. (1082.8764,69.9212) .. controls (1082.9294,77.5475) and (1082.5771,85.1645)
        .. (1081.7580,92.6406) .. controls (1080.9390,100.1166) and
        (1079.6529,107.4510) .. (1077.8800,114.5192) .. controls (1077.5196,115.9561)
        and (1077.1370,117.3814) .. (1076.7322,118.7950) .. controls
        (1076.7322,118.7950) and (1076.7322,118.7950) .. (1076.7322,118.7950) ..
        controls (1074.2802,127.3593) and (1070.9913,135.4851) .. (1066.9746,143.0895)
        .. controls (1062.9578,150.6940) and (1058.2151,157.7808) ..
        (1052.7593,164.2982) .. controls (1048.1676,169.7915) and (1043.0769,174.8553)
        .. (1037.5496,179.5517) .. controls (1032.0223,184.2481) and
        (1026.0572,188.5782) .. (1019.6764,192.5731) .. controls (1018.3554,193.4233)
        and (1016.6922,194.5197) .. (1015.0027,195.6402) .. controls
        (1013.3131,196.7606) and (1011.5970,197.9050) .. (1010.2067,198.9021) ..
        controls (1008.8163,199.8993) and (1007.7531,200.7527) .. (1007.3944,201.3169)
        .. controls (1007.0358,201.8811) and (1007.3844,202.1638) ..
        (1008.8112,201.9594) .. controls (1011.7293,201.0670) and (1015.1504,199.2333)
        .. (1018.1258,197.5435) -- cycle;

      % path57-7
      \path[fill=black] (924.5239,185.8305) .. controls (918.7582,182.1251) and
        (913.3817,177.8437) .. (908.6472,172.9140) .. controls (908.6472,172.9140) and
        (908.6472,172.9140) .. (908.6472,172.9140) .. controls (901.2639,165.2460) and
        (895.5071,156.0028) .. (891.4109,146.0842) .. controls (891.4109,146.0842) and
        (891.4109,146.0842) .. (891.4109,146.0842) .. controls (886.4868,134.1695) and
        (883.8017,121.3593) .. (882.1890,108.4463) .. controls (882.1890,108.4463) and
        (882.1890,108.4463) .. (882.1890,108.4463) .. controls (881.5242,103.1117) and
        (881.0408,97.7541) .. (880.6979,92.3810) .. controls (879.9563,80.7568) and
        (880.0073,69.0565) .. (880.4305,57.3790) .. controls (881.2586,34.4041) and
        (887.8337,11.8697) .. (897.1438,-9.2298) .. controls (897.1438,-9.2298) and
        (897.1438,-9.2298) .. (897.1438,-9.2298) .. controls (897.7713,-10.6502) and
        (898.4124,-12.0649) .. (899.0670,-13.4738) .. controls (904.0460,-24.1902) and
        (909.9962,-34.4672) .. (916.7933,-44.1124) .. controls (923.5904,-53.7575) and
        (931.2323,-62.7709) .. (939.6220,-71.0138) .. controls (939.6220,-71.0138) and
        (939.6220,-71.0138) .. (939.6220,-71.0138) .. controls (948.4531,-79.6906) and
        (958.1235,-87.4824) .. (968.5669,-94.0397) .. controls (968.5669,-94.0397) and
        (968.5669,-94.0397) .. (968.5669,-94.0397) .. controls (968.7339,-94.1446) and
        (968.9011,-94.2491) .. (969.0685,-94.3534) .. controls (979.2229,-100.6752)
        and (990.1983,-105.7052) .. (1001.6989,-108.9535) .. controls
        (1001.6989,-108.9535) and (1001.6989,-108.9535) .. (1001.6989,-108.9535) ..
        controls (1013.7073,-112.3480) and (1026.2895,-113.7719) ..
        (1038.7860,-113.1302) .. controls (1050.2798,-112.5429) and
        (1061.6807,-110.2214) .. (1072.6586,-106.5681) .. controls
        (1074.9253,-105.4696) and (1076.6604,-104.8963) .. (1077.8757,-104.7186) ..
        controls (1079.0910,-104.5410) and (1079.7832,-104.7591) ..
        (1079.9018,-105.2086) .. controls (1080.0204,-105.6580) and
        (1079.5623,-106.3399) .. (1078.4611,-107.0454) .. controls
        (1077.3599,-107.7509) and (1075.6118,-108.4809) .. (1073.1840,-108.9677) ..
        controls (1062.1209,-112.6834) and (1050.6042,-115.0689) ..
        (1038.9497,-115.7265) .. controls (1026.1681,-116.4445) and
        (1013.2868,-115.0521) .. (1000.9669,-111.6336) .. controls
        (1000.9669,-111.6336) and (1000.9669,-111.6336) .. (1000.9669,-111.6336) ..
        controls (989.4023,-108.4230) and (978.3631,-103.4708) .. (968.1353,-97.2349)
        .. controls (967.7609,-97.0066) and (967.3876,-96.7768) .. (967.0152,-96.5454)
        .. controls (967.0152,-96.5454) and (967.0152,-96.5454) .. (967.0152,-96.5454)
        .. controls (956.3221,-89.9020) and (946.4269,-82.0108) .. (937.4066,-73.2191)
        .. controls (937.4066,-73.2191) and (937.4066,-73.2191) .. (937.4066,-73.2191)
        .. controls (929.0482,-65.0725) and (921.4151,-56.1812) .. (914.6036,-46.6741)
        .. controls (907.7921,-37.1670) and (901.8040,-27.0440) .. (896.7605,-16.4842)
        .. controls (895.8355,-14.5475) and (894.9351,-12.5866) .. (894.0607,-10.6037)
        .. controls (889.4572,-0.1863) and (885.5155,10.9098) .. (882.6862,22.3460) ..
        controls (879.8569,33.7822) and (878.1474,45.5537) .. (877.8767,57.2894) ..
        controls (877.6012,69.4204) and (877.7011,81.4603) .. (878.4580,93.0944) ..
        controls (878.8017,98.3791) and (879.2500,103.5824) .. (879.8387,108.7239) ..
        controls (879.8387,108.7239) and (879.8387,108.7239) .. (879.8387,108.7239) ..
        controls (881.3575,122.0108) and (883.8504,134.9568) .. (888.6433,147.2079) ..
        controls (892.6448,157.4281) and (898.3892,167.0789) .. (906.1436,175.2626) ..
        controls (910.0815,179.4095) and (914.4909,183.1443) .. (919.2775,186.4711) ..
        controls (920.6546,187.4043) and (922.4546,188.5031) .. (924.3369,189.5060) ..
        controls (926.2193,190.5089) and (928.1821,191.4160) .. (929.8452,192.0474) ..
        controls (931.5083,192.6788) and (932.8686,193.0395) .. (933.5476,192.9928) ..
        controls (934.2267,192.9462) and (934.2202,192.5005) .. (933.2039,191.4640) ..
        controls (930.7406,189.6230) and (927.3773,187.7149) .. (924.5239,185.8305) --
        cycle;

      % path59-9
      \path[fill=black] (1143.9849,40.6691) .. controls (1142.6741,42.8900) and
        (1141.1594,45.0107) .. (1139.4907,47.0076) .. controls (1139.4907,47.0076) and
        (1139.4907,47.0076) .. (1139.4907,47.0076) .. controls (1135.4779,51.8083) and
        (1130.5771,55.9113) .. (1125.2969,59.4747) .. controls (1125.2969,59.4747) and
        (1125.2968,59.4747) .. (1125.2968,59.4747) .. controls (1112.9138,67.8321) and
        (1098.6394,73.2564) .. (1084.1194,77.5104) .. controls (1084.1194,77.5104) and
        (1084.1194,77.5104) .. (1084.1194,77.5104) .. controls (1078.9231,79.0300) and
        (1073.6779,80.3864) .. (1068.3960,81.6088) .. controls (1057.2513,84.1881) and
        (1045.9064,85.9808) .. (1034.5132,87.2656) .. controls (1018.5727,89.0618) and
        (1002.5246,89.8538) .. (986.4857,89.7823) .. controls (986.4857,89.7823) and
        (986.4857,89.7823) .. (986.4857,89.7823) .. controls (984.1995,89.7722) and
        (981.9133,89.7463) .. (979.6271,89.7060) .. controls (964.9005,89.4462) and
        (950.1984,88.0979) .. (935.5984,86.1529) .. controls (920.6581,84.1606) and
        (905.8234,81.5500) .. (891.1782,78.2434) .. controls (887.6621,77.4495) and
        (884.1601,76.5966) .. (880.6746,75.6789) .. controls (880.6746,75.6789) and
        (880.6746,75.6789) .. (880.6746,75.6789) .. controls (865.0813,71.5544) and
        (849.6060,66.4388) .. (836.3355,57.5650) .. controls (836.3355,57.5650) and
        (836.3355,57.5650) .. (836.3355,57.5650) .. controls (831.1426,54.0885) and
        (826.3210,50.0025) .. (822.7493,44.9569) .. controls (822.7493,44.9569) and
        (822.7493,44.9569) .. (822.7493,44.9569) .. controls (820.6889,42.0508) and
        (819.0889,38.7919) .. (818.1257,35.3595) .. controls (817.8999,33.3493) and
        (817.6393,31.9453) .. (817.2713,31.0183) .. controls (816.9032,30.0913) and
        (816.4304,29.6491) .. (815.9720,29.7187) .. controls (815.5137,29.7883) and
        (815.0713,30.3853) .. (814.9326,31.5057) .. controls (814.7940,32.6260) and
        (814.9691,34.2825) .. (815.8353,36.2163) .. controls (816.8581,39.8563) and
        (818.5287,43.2979) .. (820.6850,46.3706) .. controls (820.6850,46.3706) and
        (820.6850,46.3706) .. (820.6850,46.3706) .. controls (824.4827,51.7704) and
        (829.5009,56.1071) .. (834.8565,59.7152) .. controls (834.8565,59.7152) and
        (834.8565,59.7152) .. (834.8565,59.7152) .. controls (848.4837,68.9066) and
        (864.2144,74.2469) .. (879.9153,78.4615) .. controls (879.9153,78.4615) and
        (879.9153,78.4615) .. (879.9153,78.4615) .. controls (883.2438,79.3572) and
        (886.5864,80.1943) .. (889.9409,80.9775) .. controls (904.8874,84.4670) and
        (920.0137,87.2299) .. (935.2256,89.3311) .. controls (949.4375,91.2960) and
        (963.7616,92.6857) .. (978.1268,93.0428) .. controls (980.8819,93.1113) and
        (983.6619,93.1482) .. (986.4615,93.1520) .. controls (1002.2906,93.1728) and
        (1018.7224,92.0146) .. (1034.8180,89.9036) .. controls (1046.6201,88.3570) and
        (1058.2669,86.3024) .. (1069.3570,83.7281) .. controls (1074.6017,82.5106) and
        (1079.7332,81.2225) .. (1084.7780,79.8242) .. controls (1084.7780,79.8242) and
        (1084.7780,79.8242) .. (1084.7780,79.8242) .. controls (1092.2441,77.7580) and
        (1099.5671,75.4636) .. (1106.6595,72.6382) .. controls (1113.7520,69.8127) and
        (1120.6201,66.4576) .. (1127.1142,62.2376) .. controls (1132.5737,58.6894) and
        (1137.7564,54.4625) .. (1142.1050,49.2886) .. controls (1143.1728,48.0184) and
        (1144.1847,46.6881) .. (1145.1276,45.3013) .. controls (1145.8896,44.1457) and
        (1146.7570,42.6162) .. (1147.4832,41.0002) .. controls (1148.2095,39.3842) and
        (1148.7906,37.6847) .. (1149.1129,36.2492) .. controls (1149.4353,34.8136) and
        (1149.5070,33.6483) .. (1149.3212,33.0870) .. controls (1149.1354,32.5258) and
        (1148.7123,32.5749) .. (1147.9305,33.4582) .. controls (1147.3031,34.5129) and
        (1146.6871,35.7319) .. (1146.0356,36.9798) .. controls (1145.3841,38.2278) and
        (1144.6969,39.5047) .. (1143.9849,40.6691) -- cycle;

      % path61-0
      \path[fill=black,opacity=0.280] (1147.6261,16.2554) .. controls
        (1146.0780,13.8954) and (1144.3397,11.6725) .. (1142.4581,9.6003) .. controls
        (1142.4581,9.6003) and (1142.4581,9.6003) .. (1142.4581,9.6003) .. controls
        (1137.9502,4.6354) and (1132.6989,0.4762) .. (1127.1447,-3.0870) .. controls
        (1127.1447,-3.0870) and (1127.1447,-3.0870) .. (1127.1447,-3.0870) .. controls
        (1114.1495,-11.4224) and (1099.5941,-16.8581) .. (1084.8642,-20.9300) ..
        controls (1084.8642,-20.9300) and (1084.8642,-20.9300) .. (1084.8642,-20.9300)
        .. controls (1079.5895,-22.3910) and (1074.2700,-23.6714) ..
        (1068.9170,-24.7844) .. controls (1057.6221,-27.1327) and (1046.2155,-28.9249)
        .. (1034.7687,-30.2793) .. controls (1018.7525,-32.1758) and
        (1002.6114,-33.2112) .. (986.4680,-33.0517) .. controls (986.4680,-33.0517)
        and (986.4680,-33.0517) .. (986.4680,-33.0517) .. controls (984.1682,-33.0289)
        and (981.8689,-32.9837) .. (979.5704,-32.9160) .. controls (964.7646,-32.4799)
        and (949.9700,-31.5988) .. (935.2294,-30.1023) .. controls (920.1451,-28.5728)
        and (905.0656,-26.3910) .. (890.2978,-22.8016) .. controls (886.7522,-21.9398)
        and (883.2213,-21.0158) .. (879.7081,-20.0239) .. controls (879.7081,-20.0239)
        and (879.7081,-20.0239) .. (879.7081,-20.0239) .. controls (864.0239,-15.6138)
        and (848.2939,-9.9744) .. (834.8757,-0.4902) .. controls (834.8757,-0.4902)
        and (834.8757,-0.4902) .. (834.8757,-0.4902) .. controls (829.6024,3.2326) and
        (824.7041,7.6809) .. (821.1013,13.1115) .. controls (821.1013,13.1115) and
        (821.1013,13.1115) .. (821.1013,13.1115) .. controls (819.0123,16.2642) and
        (817.4467,19.7715) .. (816.5640,23.4310) .. controls (815.7529,25.4047) and
        (815.6465,27.0096) .. (815.8280,28.0412) .. controls (816.0094,29.0728) and
        (816.4713,29.5480) .. (816.9339,29.5204) .. controls (817.3965,29.4928) and
        (817.8618,28.9781) .. (818.2160,28.0342) .. controls (818.5702,27.0903) and
        (818.8126,25.7234) .. (818.9767,23.8255) .. controls (819.8157,20.5419) and
        (821.2628,17.3796) .. (823.1656,14.5252) .. controls (823.1656,14.5252) and
        (823.1656,14.5252) .. (823.1656,14.5252) .. controls (826.5425,9.4488) and
        (831.2439,5.2512) .. (836.3547,1.6600) .. controls (836.3547,1.6600) and
        (836.3547,1.6600) .. (836.3547,1.6600) .. controls (849.4165,-7.5068) and
        (864.8916,-12.9215) .. (880.4674,-17.2413) .. controls (880.4674,-17.2413) and
        (880.4674,-17.2413) .. (880.4674,-17.2413) .. controls (883.7654,-18.1536) and
        (887.0799,-19.0057) .. (890.4085,-19.8023) .. controls (905.2394,-23.3519) and
        (920.4083,-25.4588) .. (935.6022,-26.9240) .. controls (949.7972,-28.2912) and
        (964.0241,-29.0969) .. (978.2433,-29.5078) .. controls (980.9704,-29.5866) and
        (983.7215,-29.6458) .. (986.4922,-29.6820) .. controls (1002.1417,-29.8871)
        and (1018.4309,-29.2428) .. (1034.4638,-27.6414) .. controls
        (1046.2205,-26.4657) and (1057.8397,-24.7859) .. (1068.8979,-22.4574) ..
        controls (1074.1276,-21.3563) and (1079.2227,-20.0723) .. (1084.2056,-18.6162)
        .. controls (1084.2056,-18.6162) and (1084.2056,-18.6162) ..
        (1084.2056,-18.6162) .. controls (1091.5836,-16.4571) and (1098.7476,-13.9473)
        .. (1105.6223,-10.9522) .. controls (1112.4970,-7.9570) and
        (1119.0831,-4.4828) .. (1125.3274,-0.3240) .. controls (1130.5649,3.1640) and
        (1135.5357,7.1337) .. (1139.8438,11.8814) .. controls (1140.8986,13.0439) and
        (1141.9130,14.2558) .. (1142.8768,15.5167) .. controls (1143.6879,16.5454) and
        (1144.7110,17.8545) .. (1145.7136,19.2222) .. controls (1146.7162,20.5899) and
        (1147.6955,22.0176) .. (1148.5270,23.2055) .. controls (1149.3586,24.3934) and
        (1150.0509,25.3393) .. (1150.5541,25.6725) .. controls (1151.0573,26.0056) and
        (1151.3932,25.7176) .. (1151.3570,24.4411) .. controls (1151.0880,23.1292) and
        (1150.5664,21.7010) .. (1149.9010,20.2974) .. controls (1149.2357,18.8939) and
        (1148.4276,17.5161) .. (1147.6261,16.2554) -- cycle;

    \end{scope}
    % path1726
    \path[draw=black,line join=miter,line cap=butt,line width=0.800pt,-latex']
      (437.7930,855.6026) -- (373.5376,814.7108);

    % path1728
    \path[draw=black,line join=miter,line cap=butt,line width=0.800pt,-latex']
      (357.7930,863.6026) -- (293.5376,822.7108);

    % path1730
    \path[draw=black,line join=miter,line cap=butt,line width=0.800pt,-latex']
      (257.7930,851.6026) -- (193.5376,810.7108);

    % path1732
    \path[draw=black,line join=miter,line cap=butt,line width=0.800pt,-latex']
      (199.7930,829.6026) -- (135.5376,788.7108);

    % path1734
    \path[draw=black,line join=miter,line cap=butt,line width=0.800pt,-latex']
      (191.7930,787.6026) -- (127.5376,746.7108);

    % path1736
    \path[draw=black,line join=miter,line cap=butt,line width=0.800pt,-latex']
      (237.7930,757.6026) -- (173.5376,716.7108);

    % path1738
    \path[draw=black,line join=miter,line cap=butt,line width=0.800pt,-latex']
      (311.7930,743.6026) -- (247.5376,702.7108);

    % path1740
    \path[draw=black,line join=miter,line cap=butt,line width=0.800pt,-latex']
      (401.7930,743.6026) -- (337.5376,702.7108);

    % path1742
    \path[draw=black,line join=miter,line cap=butt,line width=0.800pt,-latex']
      (465.7930,757.6026) -- (401.5376,716.7108);

    % path1744
    \path[draw=black,line join=miter,line cap=butt,line width=0.800pt,-latex']
      (517.7931,805.6026) -- (453.5376,764.7108);

    % path1746
    \path[draw=black,line join=miter,line cap=butt,line width=0.800pt,-latex']
      (495.7931,833.6026) -- (431.5376,792.7108);
\begin{scope}[cm={{1.0,0.14753,0.0,1.0,(-502.54294,-1831.0605)}}]% g1970
      % rect1863
      \path[cm={{-0.50196,0.86489,0.0,1.0,(0.0,0.0)}},fill=c0000ff,opacity=0.300,nonzero
        rule,rounded corners=0.0000cm] (-2394.9067,4333.2295) rectangle
        (-2145.5801,4477.3559);

      % path1865
      \path[draw=black,fill=black,line join=miter,line cap=butt,line width=0.800pt,-latex']
        (1133.5412,2449.3650) -- (1173.5412,2380.4440);

      % path1867
      \path[draw=black,fill=black,line join=miter,line cap=butt,line width=0.800pt,-latex']
        (1133.0412,2449.7265) -- (1133.0412,2409.7265);

      % path1869
      \path[draw=cff0000,fill=cff0000,fill=cff0000,line join=miter,line cap=butt,line width=0.800pt,-latex']
        (1133.0412,2449.7265) -- (1192.5043,2308.0941);

    \end{scope}

    \begin{scope}[shift={(-502.54294,-1662.9435)}]% g1964
      % rect1873
      \path[cm={{-0.9899,0.14176,0.0,1.0,(0.0,0.0)}},fill=c0000ff,opacity=0.300,nonzero
        rule,rounded corners=0.0000cm] (-1329.2004,2516.7695) rectangle
        (-1188.5658,2655.9838);

      % path1875
      \path[draw=black,fill=black,line join=miter,line cap=butt,line width=0.800pt,-latex']
        (1247.1693,2407.7705) -- (1287.1693,2402.0421);

      % path1877
      \path[draw=black,fill=black,line join=miter,line cap=butt,line width=0.800pt,-latex']
        (1246.6693,2407.3421) -- (1246.6693,2367.3421);

      % path1879
      \path[draw=cff0000,fill=cff0000,line join=miter,line cap=butt,line width=0.800pt,-latex']
        (1246.6693,2407.3421) -- (1306.1324,2359.6504);

    \end{scope}
    \begin{scope}[shift={(-502.54294,-1662.9435)}]% g1953
      % rect1883
      \path[cm={{-0.94685,-0.32168,0.0,1.0,(0.0,0.0)}},fill=c0000ff,opacity=0.300,nonzero
        rule,rounded corners=0.0000cm] (-1547.6748,1872.0035) rectangle
        (-1400.6454,2011.2178);

      % path1885
      \path[draw=black,fill=black,line join=miter,line cap=butt,line width=0.800pt,-latex']
        (1396.8041,2416.1648) -- (1436.8041,2429.7545);

      % path1887
      \path[draw=black,fill=black,line join=miter,line cap=butt,line width=0.800pt,-latex']
        (1396.3041,2415.4950) -- (1396.3041,2375.4950);

      % path1889
      \path[draw=cff0000,fill=cff0000,fill=cff0000,line join=miter,line cap=butt,line width=0.800pt,-latex']
        (1396.3041,2415.4950) -- (1455.7672,2396.5211);

    \end{scope}

    \begin{scope}[shift={(-190.60299,773.9938)},opacity=0.500,transparency group]% g1748
      % path1750
      \path[fill=black] (814.4212,58.5435) .. controls (820.3754,93.7311) and
        (836.8215,127.0965) .. (861.8065,152.5185) .. controls (861.8065,152.5185) and
        (861.8065,152.5185) .. (861.8065,152.5185) .. controls (877.3745,168.3508) and
        (896.0728,181.0812) .. (916.5208,189.7208) .. controls (916.5208,189.7208) and
        (916.5208,189.7208) .. (916.5208,189.7208) .. controls (938.2006,198.8777) and
        (961.7638,203.3783) .. (985.2680,203.2928) .. controls (985.2680,203.2928) and
        (985.2680,203.2928) .. (985.2680,203.2928) .. controls (988.4665,203.2814) and
        (991.6647,203.1752) .. (994.8567,202.9746) .. controls (1015.0320,201.7064)
        and (1034.9702,196.7427) .. (1053.2767,188.1710) .. controls
        (1073.3535,178.7669) and (1091.3975,165.1492) .. (1106.2074,148.6496) ..
        controls (1106.2074,148.6496) and (1106.2074,148.6496) .. (1106.2074,148.6496)
        .. controls (1120.8905,132.2919) and (1132.4412,113.1366) ..
        (1140.1348,92.5338) .. controls (1147.1845,73.6587) and (1150.9849,53.5407) ..
        (1151.0552,33.3886) .. controls (1151.0652,31.5089) and (1151.0372,29.6289) ..
        (1150.9802,27.7496) .. controls (1150.9802,27.7496) and (1150.9802,27.7496) ..
        (1150.9802,27.7496) .. controls (1150.3459,6.7970) and (1145.7929,-14.0185) ..
        (1137.7201,-33.3655) .. controls (1129.6473,-52.7126) and (1118.0511,-70.5899)
        .. (1103.3264,-85.6015) .. controls (1103.3264,-85.6015) and
        (1103.3263,-85.6015) .. (1103.3263,-85.6015) .. controls (1088.9567,-100.2437)
        and (1071.7700,-112.1497) .. (1052.9265,-120.3583) .. controls
        (1052.9265,-120.3583) and (1052.9264,-120.3583) .. (1052.9264,-120.3583) ..
        controls (1034.2127,-128.5074) and (1013.9178,-132.9553) ..
        (993.5271,-133.5322) .. controls (992.5638,-133.5595) and (991.6002,-133.5783)
        .. (990.6366,-133.5886) .. controls (968.3513,-133.8241) and
        (946.1243,-129.4717) .. (925.4123,-121.4288) .. controls (904.6002,-113.3501)
        and (885.2493,-101.5604) .. (868.6019,-86.7416) .. controls
        (868.6019,-86.7416) and (868.6019,-86.7416) .. (868.6019,-86.7416) .. controls
        (852.0572,-72.0170) and (838.1847,-54.2239) .. (828.5521,-34.3115) .. controls
        (828.5521,-34.3115) and (828.5521,-34.3115) .. (828.5521,-34.3115) .. controls
        (821.1762,-19.0603) and (816.3964,-2.5704) .. (814.5363,14.2211) .. controls
        (813.5712,19.9703) and (813.4850,24.2752) .. (813.6871,27.0631) .. controls
        (813.8891,29.8509) and (814.3785,31.1348) .. (814.8395,30.9987) .. controls
        (815.3006,30.8625) and (815.7362,29.3143) .. (816.0857,26.4386) .. controls
        (816.4352,23.5630) and (816.7007,19.3626) .. (817.0351,13.8445) .. controls
        (818.9464,-2.4347) and (823.6513,-18.4133) .. (830.8336,-33.1926) .. controls
        (830.8336,-33.1926) and (830.8336,-33.1926) .. (830.8336,-33.1926) .. controls
        (840.3409,-52.7626) and (854.0447,-70.2570) .. (870.3712,-84.7362) .. controls
        (870.3712,-84.7362) and (870.3712,-84.7362) .. (870.3712,-84.7362) .. controls
        (886.8048,-99.3078) and (905.9029,-110.8894) .. (926.4191,-118.8074) ..
        controls (946.8363,-126.6852) and (968.7161,-130.9288) .. (990.5966,-130.6503)
        .. controls (991.0043,-130.6453) and (991.4120,-130.6384) ..
        (991.8197,-130.6302) .. controls (1012.3883,-130.2142) and
        (1032.8870,-125.7854) .. (1051.7088,-117.5386) .. controls
        (1051.7088,-117.5386) and (1051.7088,-117.5386) .. (1051.7088,-117.5386) ..
        controls (1070.1704,-109.4514) and (1087.0055,-97.7342) ..
        (1101.0624,-83.3551) .. controls (1115.4667,-68.6278) and (1126.8038,-51.0718)
        .. (1134.6899,-32.0835) .. controls (1142.5760,-13.0952) and
        (1147.0149,7.3240) .. (1147.6166,27.8466) .. controls (1147.6166,27.8466) and
        (1147.6166,27.8466) .. (1147.6166,27.8466) .. controls (1147.6326,28.3811) and
        (1147.6456,28.9157) .. (1147.6556,29.4504) .. controls (1148.0610,50.0902) and
        (1144.4865,71.3087) .. (1137.0911,91.3998) .. controls (1129.6869,111.5099)
        and (1118.4636,130.5046) .. (1104.0814,146.7470) .. controls
        (1104.0814,146.7470) and (1104.0814,146.7470) .. (1104.0814,146.7470) ..
        controls (1089.5700,163.1345) and (1071.8025,176.6809) .. (1052.2415,185.9599)
        .. controls (1033.5864,194.8132) and (1013.3474,199.6927) ..
        (993.3781,200.7747) .. controls (990.6689,200.9215) and (987.9644,200.9976) ..
        (985.2676,201.0040) .. controls (961.5766,201.0586) and (938.4845,196.3434) ..
        (917.5424,187.3157) .. controls (897.3796,178.6265) and (879.2041,165.9873) ..
        (864.0382,150.3361) .. controls (853.1146,139.0685) and (843.8358,126.2066) ..
        (836.3499,112.1612) .. controls (828.8641,98.1157) and (823.1671,82.8815) ..
        (819.5655,66.8410) .. controls (818.7252,63.1879) and (817.7457,58.5408) ..
        (816.9211,53.7921) .. controls (816.0965,49.0434) and (815.4294,44.1954) ..
        (814.9282,40.1943) .. controls (814.4270,36.1933) and (814.0732,33.0404) ..
        (813.7050,31.7033) .. controls (813.5209,31.0348) and (813.3320,30.8205) ..
        (813.1355,31.1839) .. controls (812.9389,31.5473) and (812.7330,32.4884) ..
        (812.5534,34.1326) .. controls (812.4051,37.7311) and (812.5169,41.8343) ..
        (812.8565,46.0352) .. controls (813.1964,50.2361) and (813.7641,54.5330) ..
        (814.4212,58.5435) -- cycle;

      % path1752
      \path[fill=black] (1018.1258,197.5435) .. controls (1032.1810,189.3284) and
        (1044.9190,178.8829) .. (1055.1265,166.2307) .. controls (1066.2029,152.4809)
        and (1074.1578,136.3667) .. (1078.9331,119.4360) .. controls
        (1078.9331,119.4360) and (1078.9331,119.4360) .. (1078.9331,119.4360) ..
        controls (1079.3917,117.8099) and (1079.8224,116.1770) .. (1080.2257,114.5379)
        .. controls (1083.8144,99.9537) and (1085.4669,84.9341) .. (1085.5968,69.9374)
        .. controls (1085.7286,54.9173) and (1084.3602,39.8820) .. (1081.5801,25.1175)
        .. controls (1081.5801,25.1175) and (1081.5801,25.1175) .. (1081.5801,25.1175)
        .. controls (1080.5850,19.8377) and (1079.3114,14.6203) .. (1077.7951,9.4756)
        .. controls (1074.5536,-1.5229) and (1070.4034,-12.2558) ..
        (1065.6350,-22.6709) .. controls (1065.6350,-22.6709) and (1065.6350,-22.6709)
        .. (1065.6350,-22.6709) .. controls (1057.3777,-40.7226) and
        (1046.9982,-57.9471) .. (1033.9537,-73.0247) .. controls (1031.2633,-76.1370)
        and (1028.4516,-79.1565) .. (1025.5200,-82.0570) .. controls
        (1013.8830,-93.5707) and (1000.5506,-103.4031) .. (985.6962,-110.2445) ..
        controls (985.6962,-110.2445) and (985.6962,-110.2445) .. (985.6962,-110.2445)
        .. controls (976.2165,-114.6053) and (966.1590,-117.6954) ..
        (955.8750,-119.2930) .. controls (947.1619,-120.6449) and (938.3137,-120.9197)
        .. (929.5634,-120.2180) .. controls (927.1037,-120.3482) and
        (925.3145,-120.0704) .. (924.1845,-119.6647) .. controls (923.0544,-119.2591)
        and (922.5787,-118.7250) .. (922.6920,-118.2745) .. controls
        (922.8053,-117.8240) and (923.5041,-117.4560) .. (924.7323,-117.3308) ..
        controls (925.9605,-117.2057) and (927.7151,-117.3228) .. (929.9989,-117.8028)
        .. controls (938.4820,-118.4245) and (947.0413,-118.1094) ..
        (955.4496,-116.7634) .. controls (965.4833,-115.1594) and (975.2953,-112.0924)
        .. (984.5377,-107.7900) .. controls (984.5377,-107.7900) and
        (984.5377,-107.7900) .. (984.5377,-107.7900) .. controls (998.8274,-101.1442)
        and (1011.6802,-91.6126) .. (1022.9499,-80.4534) .. controls
        (1025.9904,-77.4427) and (1028.9018,-74.2995) .. (1031.6828,-71.0549) ..
        controls (1044.4532,-56.1489) and (1054.5918,-39.1165) .. (1062.6757,-21.2767)
        .. controls (1062.6757,-21.2767) and (1062.6757,-21.2767) ..
        (1062.6757,-21.2767) .. controls (1067.1138,-11.4699) and (1071.0008,-1.4168)
        .. (1074.0909,8.8347) .. controls (1075.7460,14.3254) and (1077.1804,19.9663)
        .. (1078.3065,25.7332) .. controls (1081.1101,40.0716) and (1082.7695,55.0219)
        .. (1082.8764,69.9212) .. controls (1082.9294,77.5475) and (1082.5771,85.1645)
        .. (1081.7580,92.6406) .. controls (1080.9390,100.1166) and
        (1079.6529,107.4510) .. (1077.8800,114.5192) .. controls (1077.5196,115.9561)
        and (1077.1370,117.3814) .. (1076.7322,118.7950) .. controls
        (1076.7322,118.7950) and (1076.7322,118.7950) .. (1076.7322,118.7950) ..
        controls (1074.2802,127.3593) and (1070.9913,135.4851) .. (1066.9746,143.0895)
        .. controls (1062.9578,150.6940) and (1058.2151,157.7808) ..
        (1052.7593,164.2982) .. controls (1048.1676,169.7915) and (1043.0769,174.8553)
        .. (1037.5496,179.5517) .. controls (1032.0223,184.2481) and
        (1026.0572,188.5782) .. (1019.6764,192.5731) .. controls (1018.3554,193.4233)
        and (1016.6922,194.5197) .. (1015.0027,195.6402) .. controls
        (1013.3131,196.7606) and (1011.5970,197.9050) .. (1010.2067,198.9021) ..
        controls (1008.8163,199.8993) and (1007.7531,200.7527) .. (1007.3944,201.3169)
        .. controls (1007.0358,201.8811) and (1007.3844,202.1638) ..
        (1008.8112,201.9594) .. controls (1011.7293,201.0670) and (1015.1504,199.2333)
        .. (1018.1258,197.5435) -- cycle;

      % path1754
      \path[fill=black] (924.5239,185.8305) .. controls (918.7582,182.1251) and
        (913.3817,177.8437) .. (908.6472,172.9140) .. controls (908.6472,172.9140) and
        (908.6472,172.9140) .. (908.6472,172.9140) .. controls (901.2639,165.2460) and
        (895.5071,156.0028) .. (891.4109,146.0842) .. controls (891.4109,146.0842) and
        (891.4109,146.0842) .. (891.4109,146.0842) .. controls (886.4868,134.1695) and
        (883.8017,121.3593) .. (882.1890,108.4463) .. controls (882.1890,108.4463) and
        (882.1890,108.4463) .. (882.1890,108.4463) .. controls (881.5242,103.1117) and
        (881.0408,97.7541) .. (880.6979,92.3810) .. controls (879.9563,80.7568) and
        (880.0073,69.0565) .. (880.4305,57.3790) .. controls (881.2586,34.4041) and
        (887.8337,11.8697) .. (897.1438,-9.2298) .. controls (897.1438,-9.2298) and
        (897.1438,-9.2298) .. (897.1438,-9.2298) .. controls (897.7713,-10.6502) and
        (898.4124,-12.0649) .. (899.0670,-13.4738) .. controls (904.0460,-24.1902) and
        (909.9962,-34.4672) .. (916.7933,-44.1124) .. controls (923.5904,-53.7575) and
        (931.2323,-62.7709) .. (939.6220,-71.0138) .. controls (939.6220,-71.0138) and
        (939.6220,-71.0138) .. (939.6220,-71.0138) .. controls (948.4531,-79.6906) and
        (958.1235,-87.4824) .. (968.5669,-94.0397) .. controls (968.5669,-94.0397) and
        (968.5669,-94.0397) .. (968.5669,-94.0397) .. controls (968.7339,-94.1446) and
        (968.9011,-94.2491) .. (969.0685,-94.3534) .. controls (979.2229,-100.6752)
        and (990.1983,-105.7052) .. (1001.6989,-108.9535) .. controls
        (1001.6989,-108.9535) and (1001.6989,-108.9535) .. (1001.6989,-108.9535) ..
        controls (1013.7073,-112.3480) and (1026.2895,-113.7719) ..
        (1038.7860,-113.1302) .. controls (1050.2798,-112.5429) and
        (1061.6807,-110.2214) .. (1072.6586,-106.5681) .. controls
        (1074.9253,-105.4696) and (1076.6604,-104.8963) .. (1077.8757,-104.7186) ..
        controls (1079.0910,-104.5410) and (1079.7832,-104.7591) ..
        (1079.9018,-105.2086) .. controls (1080.0204,-105.6580) and
        (1079.5623,-106.3399) .. (1078.4611,-107.0454) .. controls
        (1077.3599,-107.7509) and (1075.6118,-108.4809) .. (1073.1840,-108.9677) ..
        controls (1062.1209,-112.6834) and (1050.6042,-115.0689) ..
        (1038.9497,-115.7265) .. controls (1026.1681,-116.4445) and
        (1013.2868,-115.0521) .. (1000.9669,-111.6336) .. controls
        (1000.9669,-111.6336) and (1000.9669,-111.6336) .. (1000.9669,-111.6336) ..
        controls (989.4023,-108.4230) and (978.3631,-103.4708) .. (968.1353,-97.2349)
        .. controls (967.7609,-97.0066) and (967.3876,-96.7768) .. (967.0152,-96.5454)
        .. controls (967.0152,-96.5454) and (967.0152,-96.5454) .. (967.0152,-96.5454)
        .. controls (956.3221,-89.9020) and (946.4269,-82.0108) .. (937.4066,-73.2191)
        .. controls (937.4066,-73.2191) and (937.4066,-73.2191) .. (937.4066,-73.2191)
        .. controls (929.0482,-65.0725) and (921.4151,-56.1812) .. (914.6036,-46.6741)
        .. controls (907.7921,-37.1670) and (901.8040,-27.0440) .. (896.7605,-16.4842)
        .. controls (895.8355,-14.5475) and (894.9351,-12.5866) .. (894.0607,-10.6037)
        .. controls (889.4572,-0.1863) and (885.5155,10.9098) .. (882.6862,22.3460) ..
        controls (879.8569,33.7822) and (878.1474,45.5537) .. (877.8767,57.2894) ..
        controls (877.6012,69.4204) and (877.7011,81.4603) .. (878.4580,93.0944) ..
        controls (878.8017,98.3791) and (879.2500,103.5824) .. (879.8387,108.7239) ..
        controls (879.8387,108.7239) and (879.8387,108.7239) .. (879.8387,108.7239) ..
        controls (881.3575,122.0108) and (883.8504,134.9568) .. (888.6433,147.2079) ..
        controls (892.6448,157.4281) and (898.3892,167.0789) .. (906.1436,175.2626) ..
        controls (910.0815,179.4095) and (914.4909,183.1443) .. (919.2775,186.4711) ..
        controls (920.6546,187.4043) and (922.4546,188.5031) .. (924.3369,189.5060) ..
        controls (926.2193,190.5089) and (928.1821,191.4160) .. (929.8452,192.0474) ..
        controls (931.5083,192.6788) and (932.8686,193.0395) .. (933.5476,192.9928) ..
        controls (934.2267,192.9462) and (934.2202,192.5005) .. (933.2039,191.4640) ..
        controls (930.7406,189.6230) and (927.3773,187.7149) .. (924.5239,185.8305) --
        cycle;

      % path1756
      \path[fill=black] (1143.9849,40.6691) .. controls (1142.6741,42.8900) and
        (1141.1594,45.0107) .. (1139.4907,47.0076) .. controls (1139.4907,47.0076) and
        (1139.4907,47.0076) .. (1139.4907,47.0076) .. controls (1135.4779,51.8083) and
        (1130.5771,55.9113) .. (1125.2969,59.4747) .. controls (1125.2969,59.4747) and
        (1125.2968,59.4747) .. (1125.2968,59.4747) .. controls (1112.9138,67.8321) and
        (1098.6394,73.2564) .. (1084.1194,77.5104) .. controls (1084.1194,77.5104) and
        (1084.1194,77.5104) .. (1084.1194,77.5104) .. controls (1078.9231,79.0300) and
        (1073.6779,80.3864) .. (1068.3960,81.6088) .. controls (1057.2513,84.1881) and
        (1045.9064,85.9808) .. (1034.5132,87.2656) .. controls (1018.5727,89.0618) and
        (1002.5246,89.8538) .. (986.4857,89.7823) .. controls (986.4857,89.7823) and
        (986.4857,89.7823) .. (986.4857,89.7823) .. controls (984.1995,89.7722) and
        (981.9133,89.7463) .. (979.6271,89.7060) .. controls (964.9005,89.4462) and
        (950.1984,88.0979) .. (935.5984,86.1529) .. controls (920.6581,84.1606) and
        (905.8234,81.5500) .. (891.1782,78.2434) .. controls (887.6621,77.4495) and
        (884.1601,76.5966) .. (880.6746,75.6789) .. controls (880.6746,75.6789) and
        (880.6746,75.6789) .. (880.6746,75.6789) .. controls (865.0813,71.5544) and
        (849.6060,66.4388) .. (836.3355,57.5650) .. controls (836.3355,57.5650) and
        (836.3355,57.5650) .. (836.3355,57.5650) .. controls (831.1426,54.0885) and
        (826.3210,50.0025) .. (822.7493,44.9569) .. controls (822.7493,44.9569) and
        (822.7493,44.9569) .. (822.7493,44.9569) .. controls (820.6889,42.0508) and
        (819.0889,38.7919) .. (818.1257,35.3595) .. controls (817.8999,33.3493) and
        (817.6393,31.9453) .. (817.2713,31.0183) .. controls (816.9032,30.0913) and
        (816.4304,29.6491) .. (815.9720,29.7187) .. controls (815.5137,29.7883) and
        (815.0713,30.3853) .. (814.9326,31.5057) .. controls (814.7940,32.6260) and
        (814.9691,34.2825) .. (815.8353,36.2163) .. controls (816.8581,39.8563) and
        (818.5287,43.2979) .. (820.6850,46.3706) .. controls (820.6850,46.3706) and
        (820.6850,46.3706) .. (820.6850,46.3706) .. controls (824.4827,51.7704) and
        (829.5009,56.1071) .. (834.8565,59.7152) .. controls (834.8565,59.7152) and
        (834.8565,59.7152) .. (834.8565,59.7152) .. controls (848.4837,68.9066) and
        (864.2144,74.2469) .. (879.9153,78.4615) .. controls (879.9153,78.4615) and
        (879.9153,78.4615) .. (879.9153,78.4615) .. controls (883.2438,79.3572) and
        (886.5864,80.1943) .. (889.9409,80.9775) .. controls (904.8874,84.4670) and
        (920.0137,87.2299) .. (935.2256,89.3311) .. controls (949.4375,91.2960) and
        (963.7616,92.6857) .. (978.1268,93.0428) .. controls (980.8819,93.1113) and
        (983.6619,93.1482) .. (986.4615,93.1520) .. controls (1002.2906,93.1728) and
        (1018.7224,92.0146) .. (1034.8180,89.9036) .. controls (1046.6201,88.3570) and
        (1058.2669,86.3024) .. (1069.3570,83.7281) .. controls (1074.6017,82.5106) and
        (1079.7332,81.2225) .. (1084.7780,79.8242) .. controls (1084.7780,79.8242) and
        (1084.7780,79.8242) .. (1084.7780,79.8242) .. controls (1092.2441,77.7580) and
        (1099.5671,75.4636) .. (1106.6595,72.6382) .. controls (1113.7520,69.8127) and
        (1120.6201,66.4576) .. (1127.1142,62.2376) .. controls (1132.5737,58.6894) and
        (1137.7564,54.4625) .. (1142.1050,49.2886) .. controls (1143.1728,48.0184) and
        (1144.1847,46.6881) .. (1145.1276,45.3013) .. controls (1145.8896,44.1457) and
        (1146.7570,42.6162) .. (1147.4832,41.0002) .. controls (1148.2095,39.3842) and
        (1148.7906,37.6847) .. (1149.1129,36.2492) .. controls (1149.4353,34.8136) and
        (1149.5070,33.6483) .. (1149.3212,33.0870) .. controls (1149.1354,32.5258) and
        (1148.7123,32.5749) .. (1147.9305,33.4582) .. controls (1147.3031,34.5129) and
        (1146.6871,35.7319) .. (1146.0356,36.9798) .. controls (1145.3841,38.2278) and
        (1144.6969,39.5047) .. (1143.9849,40.6691) -- cycle;

      % path1758
      \path[fill=black,opacity=0.280] (1147.6261,16.2554) .. controls
        (1146.0780,13.8954) and (1144.3397,11.6725) .. (1142.4581,9.6003) .. controls
        (1142.4581,9.6003) and (1142.4581,9.6003) .. (1142.4581,9.6003) .. controls
        (1137.9502,4.6354) and (1132.6989,0.4762) .. (1127.1447,-3.0870) .. controls
        (1127.1447,-3.0870) and (1127.1447,-3.0870) .. (1127.1447,-3.0870) .. controls
        (1114.1495,-11.4224) and (1099.5941,-16.8581) .. (1084.8642,-20.9300) ..
        controls (1084.8642,-20.9300) and (1084.8642,-20.9300) .. (1084.8642,-20.9300)
        .. controls (1079.5895,-22.3910) and (1074.2700,-23.6714) ..
        (1068.9170,-24.7844) .. controls (1057.6221,-27.1327) and (1046.2155,-28.9249)
        .. (1034.7687,-30.2793) .. controls (1018.7525,-32.1758) and
        (1002.6114,-33.2112) .. (986.4680,-33.0517) .. controls (986.4680,-33.0517)
        and (986.4680,-33.0517) .. (986.4680,-33.0517) .. controls (984.1682,-33.0289)
        and (981.8689,-32.9837) .. (979.5704,-32.9160) .. controls (964.7646,-32.4799)
        and (949.9700,-31.5988) .. (935.2294,-30.1023) .. controls (920.1451,-28.5728)
        and (905.0656,-26.3910) .. (890.2978,-22.8016) .. controls (886.7522,-21.9398)
        and (883.2213,-21.0158) .. (879.7081,-20.0239) .. controls (879.7081,-20.0239)
        and (879.7081,-20.0239) .. (879.7081,-20.0239) .. controls (864.0239,-15.6138)
        and (848.2939,-9.9744) .. (834.8757,-0.4902) .. controls (834.8757,-0.4902)
        and (834.8757,-0.4902) .. (834.8757,-0.4902) .. controls (829.6024,3.2326) and
        (824.7041,7.6809) .. (821.1013,13.1115) .. controls (821.1013,13.1115) and
        (821.1013,13.1115) .. (821.1013,13.1115) .. controls (819.0123,16.2642) and
        (817.4467,19.7715) .. (816.5640,23.4310) .. controls (815.7529,25.4047) and
        (815.6465,27.0096) .. (815.8280,28.0412) .. controls (816.0094,29.0728) and
        (816.4713,29.5480) .. (816.9339,29.5204) .. controls (817.3965,29.4928) and
        (817.8618,28.9781) .. (818.2160,28.0342) .. controls (818.5702,27.0903) and
        (818.8126,25.7234) .. (818.9767,23.8255) .. controls (819.8157,20.5419) and
        (821.2628,17.3796) .. (823.1656,14.5252) .. controls (823.1656,14.5252) and
        (823.1656,14.5252) .. (823.1656,14.5252) .. controls (826.5425,9.4488) and
        (831.2439,5.2512) .. (836.3547,1.6600) .. controls (836.3547,1.6600) and
        (836.3547,1.6600) .. (836.3547,1.6600) .. controls (849.4165,-7.5068) and
        (864.8916,-12.9215) .. (880.4674,-17.2413) .. controls (880.4674,-17.2413) and
        (880.4674,-17.2413) .. (880.4674,-17.2413) .. controls (883.7654,-18.1536) and
        (887.0799,-19.0057) .. (890.4085,-19.8023) .. controls (905.2394,-23.3519) and
        (920.4083,-25.4588) .. (935.6022,-26.9240) .. controls (949.7972,-28.2912) and
        (964.0241,-29.0969) .. (978.2433,-29.5078) .. controls (980.9704,-29.5866) and
        (983.7215,-29.6458) .. (986.4922,-29.6820) .. controls (1002.1417,-29.8871)
        and (1018.4309,-29.2428) .. (1034.4638,-27.6414) .. controls
        (1046.2205,-26.4657) and (1057.8397,-24.7859) .. (1068.8979,-22.4574) ..
        controls (1074.1276,-21.3563) and (1079.2227,-20.0723) .. (1084.2056,-18.6162)
        .. controls (1084.2056,-18.6162) and (1084.2056,-18.6162) ..
        (1084.2056,-18.6162) .. controls (1091.5836,-16.4571) and (1098.7476,-13.9473)
        .. (1105.6223,-10.9522) .. controls (1112.4970,-7.9570) and
        (1119.0831,-4.4828) .. (1125.3274,-0.3240) .. controls (1130.5649,3.1640) and
        (1135.5357,7.1337) .. (1139.8438,11.8814) .. controls (1140.8986,13.0439) and
        (1141.9130,14.2558) .. (1142.8768,15.5167) .. controls (1143.6879,16.5454) and
        (1144.7110,17.8545) .. (1145.7136,19.2222) .. controls (1146.7162,20.5899) and
        (1147.6955,22.0176) .. (1148.5270,23.2055) .. controls (1149.3586,24.3934) and
        (1150.0509,25.3393) .. (1150.5541,25.6725) .. controls (1151.0573,26.0056) and
        (1151.3932,25.7176) .. (1151.3570,24.4411) .. controls (1151.0880,23.1292) and
        (1150.5664,21.7010) .. (1149.9010,20.2974) .. controls (1149.2357,18.8939) and
        (1148.4276,17.5161) .. (1147.6261,16.2554) -- cycle;

    \end{scope}
    \begin{scope}[shift={(-502.54294,-1660.9435)}]% g1987
      % rect1807
      \path[fill=c0000ff,opacity=0.300,nonzero rule,rounded corners=0.0000cm]
        (1250.3192,2456.6782) rectangle (1389.5335,2595.8925);

      \begin{scope}% g1982
        % path1809
        \path[draw=black,fill=black,line join=miter,line cap=butt,line width=0.800pt,-latex']
          (1318.9263,2526.2854) -- (1278.9263,2526.2854);

        % path1811
        \path[draw=black,fill=black,line join=miter,line cap=butt,line width=0.800pt,-latex']
          (1319.4263,2525.7854) -- (1319.4263,2485.7854);

        % path1760
        \path[draw=cff0000,fill=cff0000,fill=cff0000,line join=miter,line cap=butt,line width=0.800pt,-latex']
          (1319.4263,2525.7854) -- (1259.9632,2486.6094);

      \end{scope}
    \end{scope}
    \begin{scope}[cm={{1.0,-0.02604,0.0,1.0,(-513.04658,-1631.9281)}}]% g1976
      % rect1853
      \path[cm={{0.97515,0.22156,0.0,1.0,(0.0,0.0)}},fill=c0000ff,opacity=0.300,nonzero
        rule,rounded corners=0.0000cm] (1163.4265,2171.9875) rectangle
        (1306.1890,2311.2018);

      % path1855
      \path[draw=black,fill=black,line join=miter,line cap=butt,line width=0.800pt,-latex']
        (1203.1182,2514.9535) -- (1163.1182,2505.8652);

      % path1857
      \path[draw=black,fill=black,line join=miter,line cap=butt,line width=0.800pt,-latex']
        (1203.6182,2514.5671) -- (1203.6182,2474.5671);

      % path1859
      \path[draw=cff0000,fill=cff0000,fill=cff0000,line join=miter,line cap=butt,line width=0.800pt,-latex']
        (1203.6182,2514.5671) -- (1144.1551,2461.8806);

    \end{scope}
    

    \begin{scope}[shift={(-502.54294,-1660.9435)}]% g1947
      % rect1893
      \path[cm={{0.62802,-0.7782,0.0,1.0,(0.0,0.0)}},fill=c0000ff,opacity=0.300,nonzero
        rule,rounded corners=0.0000cm] (2201.8750,4213.7412) rectangle
        (2423.5464,4352.9555);

      % path1895
      \path[draw=black,fill=black,line join=miter,line cap=butt,line width=0.800pt,-latex']
        (1451.4306,2484.8443) -- (1411.4306,2534.4093);

      % path1897
      \path[draw=black,fill=black,line join=miter,line cap=butt,line width=0.800pt,-latex']
        (1451.9306,2483.7247) -- (1451.9306,2443.7247);

      % path1899
      \path[draw=cff0000,fill=cff0000,fill=cff0000,line join=miter,line cap=butt,line width=0.800pt,-latex']
        (1451.9306,2483.7247) -- (1392.4675,2518.2309);

    \end{scope}



  \end{scope}
\end{scope}

\end{tikzpicture}


\bigskip Które wektory są wzajemnie równoległe?
\end{center}

\end{frame}

%%%%%%next-slide%%%%%
Niech $\alpha(t)\subset M$ będzie krzywą na powierzchni $M\subset \R^3$. Z 
punktu widzenia obserwatora w $\R^3$ równoległe wektory wzdłuż krzywej będą 
miały \textit{proporcjonalne współrzędne w $\R^3$} (tj. będą tworzyć pole 
wektorowe wzdłuż $\alpha$ skierowane wzdłuż jednej prostej w $\R^3$). Z punktu 
widzenia obserwatora na powierzchni, poruszającego się wzdłuż krzywej, 
równoległe będą te wektory, które będą miały \textit{proporcjonalne współrzędne 
w bazie $\{x_1, x_2\}$ w przestrzeniach stycznych $T_{\alpha(t)}M$} dla 
wszystkich $t$.

\begin{frame}[<+->]
\begin{uwaga}
Równoległość wektorów w różnych przestrzeniach stycznych do powierzchni $M$ oznacza \textbf{proporcjonalność współczynników} ich przedstawienia w standardowej bazie obu przestrzeni stycznych.
\end{uwaga}




\end{frame}
\mode<all>{\lowsection{Pochodna kierunkowa pola wektorowego}}
%%%%%%next-slide%%%%%

Przypomnijmy materiał którym zajmowaliśmy się podczas badania własności izometrii.
\begin{frame}[<+->]{Przypomnienie}

\begin{definicja}
\textbf{Gładkie pole wektorowe} na $M$ to odwzorowanie gładkie \[F\colon M\to \R^3.\]
Gładkie pole wektorowe nazywamy \textbf{stycznym} jeśli $F(p)\in T_pM\subset \R^3$ dla wszystkich $p\in M$.
%Będziemy rozważać najczęściej \[F\define \overline{F}\circ x=(F_1,F_2,F_3)\colon U\to \R^3\]
\end{definicja}
\begin{definicja}
\textbf{Pochodna kierunkowa pola wektorowego} $F$ w kierunku wektora $v\in T_p M$ jest zdefiniowana naturalnie jako
\[\nabla_v F\define(\nabla_v F_1,\nabla_vF_2,\nabla_vF_3)\] gdzie $F_i$ są funkcjami współrzędnymi pola $F$.
\end{definicja}

\end{frame}
%%%%%%next-slide%%%%%
\begin{frame}

\begin{przyklad}
Zauważmy, że wektor normalny jako odwzorowanie
\[\widehat{n}\colon M\to\R^3,\]
jest tak naprawdę gładkim polem wektorowym na $M$. Jego pochodną kierunkową 
jest...  odwzorowanie Weingartena (z odpowiednim znakiem).
\end{przyklad}


\pause \begin{lemat}
Dla pochodnych kierunkowych pól wektorowych zachodzą takie same własności jak 
dla pochodnych kierunkowych funkcji, \mode<presentation>{tj. liniowość ze 
względu na dodawanie wektorów i pól wektorowych, jednorodność, oraz 
\textit{reguła Leibniza}: \[\nabla_v(FG)=G\nabla_vF+F\nabla_vG,\]}
\mode<article>{\begin{itemize}
\pause\item $\nabla_{av+bw}F=a\nabla_v F+b\nabla_w F$
\pause\item $\nabla_v(aF+bG)=a\nabla_v F+b\nabla_vG$
\pause\item $\nabla_v(FG)=G\nabla_vF+F\nabla_vG$
\end{itemize}} \pause czyli w każdym punkcie $p$ mamy 
\[\nabla_v\langle F,G\rangle(p)=\langle\nabla_vF,G(p)\rangle+\langle F(p),\nabla_vG\rangle.\]
\end{lemat}
 
\end{frame}
%%%%%%next-slide%%%%%
\begin{frame}
 Rozważmy styczne pole wektorowe $Z\colon M\to \R^3$. Niech $p\in M$ i $v\in T_p M$. 

\pause Oczywiście nie ma powodu, aby wektor $\nabla_v Z$ należał do przestrzeni stycznej (dlaczego nie było tego problemu gdy rozważaliśmy pochodną odwzorowania Weingartena?). \pause Spróbujemy ``naprawić'' tę sytuację w brutalny sposób: rzutując ortogonalnie $\nabla_v Z$ na $T_p M$. %Powody poniższej definicji staną się jasne później.

\pause
 \begin{definicja}
 Niech $M\subset \R^3$ będzie gładką powierzchnią, oraz niech $Z\colon M\to \R^3$ będzie stycznym polem wektorowym. Niech $p\in M$ oraz $v\in T_p(M)$. \textbf{Pochodną kowariantną} definiujemy jako
 \[\widehat\nabla_v(Z)\define\Pi_{T_pM}\left(\nabla_vZ\right),\]gdzie $\Pi_W$ oznacza rzut ortogonalny na podprzestrzeń $W$.
 \end{definicja}
 \end{frame}
%%%%%%next-slide%%%%%
\begin{frame}

\begin{lemat}
  Niech $M\subset \R^3$ będzie gładką powierzchnią, oraz niech $Y, Z\colon M\to \R^3$ będą stycznymi polami wektorowymi. Niech $f\colon M\to \R$ będzie gładką funkcją, $p\in M$ punktem na $M$, $v,w\in T_p(M)$ wektorami z przestrzeni stycznej i wreszcie $a,b\in \R$ liczbami rzeczywistymi. Wtedy
 \begin{itemize}
  \item Odwzorowanie \[\widehat{\nabla}_{(\cdot)}Z\colon T_pM\to T_pM\] zadana przez $v\mapsto \widehat{\nabla}_vZ$ jest odwzorowaniem liniowym.
  \item $\widehat{\nabla}_v(Y+Z)=\widehat{\nabla}_vY+\widehat{\nabla}_v Z$.
  \item $\widehat{\nabla}_vfZ=(\widehat{\nabla}_v f)Z(p)+f(p)(\widehat{\nabla}_vZ)$.
 \item $\widehat{\nabla}_v\langle Y,Z\rangle=\langle\widehat{\nabla}_v Y, Z(p)\rangle+\langle Y(p),\widehat{\nabla}_v Z\rangle$
\end{itemize}
\end{lemat}

\pause \textcolor{ared}{\textbf{Dowód: }}\\
\mode<presentation>{Dowody tych własności są analogiczne jak własności pochodnej kierunkowej \footnotesize (wystarczy skorzystać z liniowości rzutowania na podprzestrzeń)\normalsize.} 
\mode<article>{Dowody tych własności są analogiczne jak własności pochodnej kierunkowej (lemat \ref{lem:wl-pochodnej-kierunkowej}). Wystarczy skorzystać z liniowości rzutowania ortogonalnego na podprzestrzeń.}
\hfill $\square$
\end{frame}
%%%%%%next-slide%%%%%

Zanim wprowadzimy definicję geodezyjnej potrzebujemy jeszcze jednego oznaczenia. 
Będą nas interesowały styczne pola wektorowe wzdłuż krzywych, więc mając daną 
parametryzację krzywej będzie można bardzo prosto policzyć pochodną 
kowariantną. 
\begin{frame}


Niech $\alpha\colon (a,b)\to M\subset\R^3$ będzie krzywą na powierzchni i niech $Z\colon M\to \R^3$ będzie gładkim polem wektorowym wzdłuż $\alpha$. Oznaczmy 
\[\frac{\widehat{D}(Z\circ \alpha)}{dt}(t)\define\Pi_{T_{\alpha(t)}M}\left(\frac{d (Z\circ \alpha)}{dt}(t)\right)=\widehat{\nabla}_{\alpha'(t)}Z\]


\mode<presentation>{\pause \begin{uwaga}
Zauważmy, że styczne pole wektorowe $\alpha'$ do krzywej $\alpha\subset M$ jest 
w postaci $Z\circ \alpha$. %ponieważ $\alpha'\colon (a,b)\to \R^3$.
\end{uwaga}}
\end{frame}
\begin{uwaga}
Zauważmy, że styczne pole wektorowe $\alpha'$ do krzywej $\alpha\subset M$ jest 
w postaci $Z\circ \alpha$. Będzie tak za każdym razem, kiedy pole wektorowe 
wzdłuż krzywej $\gamma$ (jak np. pole wektor\'ow stycznych $\gamma'$) jest 
zadane jako wektor zależny od zmiennej $t$ (musimy tylko zadbać, by pole było 
wystarczająco gładkie).
\end{uwaga}

%%%%%%next-slide%%%%%
\mode<all>{\midsection{Definicja geodezyjnych}}
Teraz korzystając z języka pochodnych kowariantynych możemy wypowiedzieć co to znaczy, że krzywa ``nie skręca w prawo bądź w lewo''. 
\begin{frame}

\begin{definicja}
Niech $M\subset \R^3$ będzie powierzchnią gładką i niech $\gamma\colon (a,b)\to M$ będzie krzywą na tej powierzchni. Krzywą $\gamma$ nazywamy \textbf{krzywą geodezyjną}, lub po prostu \textbf{geodezyjną} na $M$ jeśli wektory $\gamma'(t)\in T_{\gamma(t)}M$ wzdłuż krzywej $\gamma$ pozostają względem siebie równoległe\pause, tj. jeśli istnieje pole wektorowe $Z\colon M\to \R^3$ takie, że $Z\circ \gamma(t)=\gamma'(t)$, oraz \[\frac{\widehat{D}\le(Z\circ\gamma\ri)(t)}{dt}=\Pi_{T_{\alpha(t)}M}\left(\frac{d \gamma'(t)}{dt}\right)= \widehat{\nabla}_{\gamma'(t)}Z=0.\]
% gdzie $\overline{\gamma'}\colon M\to \R^3$ jest pewnym gładkim \textbf{polem wektorowym na $M$} które jest równe $\gamma'$ na obrazie krzywej $\gamma$.%w punkcie $\gamma(t)$ przyjmuje wartość $\gamma'(t)$. 
\end{definicja}

\pause\begin{uwaga}
W powyższej definicji wystarczy, żeby to pole wektorowe istniało w pewnym otoczeniu każdego punktu $\gamma(t)$.
\end{uwaga}

\end{frame}
%%%%%%next-slide%%%%%
\begin{frame}

\mode<article>{Zanalizujmy dwa proste przypadki -- płaszczyznę i sferę.}
\begin{przyklad} 
Pokażemy, że proste i tylko proste spełniają ten warunek na płaszczyźnie. \pause Załóżmy, że krzywa $\gamma$ jest geodezyjną i leży na płaszczyźnie. Wtedy oczywiście torsja $\tau_\gamma\equiv0$ oraz wektor binormalny jest stały, więc krzywa leży w płaszczyźnie ściśle stycznej (rozpiętej przez $T_\gamma$ i $N_\gamma$). \pause Wobec tego pochodna pola stycznego $\gamma'$ wzdłuż $\gamma$ będzie równa
\mode<presentation>{\[0=\frac{\widehat{D}\gamma'(t)}{dt}=\Pi_{\langle T_\gamma,N_\gamma\rangle}\left(\frac{d\gamma'(t)}{dt}\right)=\frac{d^{\,2}\gamma(t)}{dt^{\,2}}.\]\pause}
\mode<article>{\[0=\frac{\widehat{D}\gamma'(t)}{dt}=\Pi_{\langle T_\gamma,N_\gamma\rangle}\left(\frac{d\gamma'(t)}{dt}\right)=\frac{d^2\gamma(t)}{dt^2}.\]\pause}
Zatem $\gamma(t)=vt+w$ dla pewnych $v,w\in \R^2$.

\pause Z drugiej strony każda prosta na płaszczyźnie ma taką parametryzację, więc spełnione jest powyższe równanie.
\end{przyklad}

\end{frame}
%%%%%%next-slide%%%%%
\begin{frame}

\begin{uwaga}[Bardzo ważna!]
Sprzecznie z naszą intuicją, bycie krzywą geodezyjną jest własnością \textbf{parametryzacji}, a nie samej krzywej na powierzchni. Powyższy przykład udowadnia jedynie, że prosta \textit{w postaci parametrycznej} jest geodezyjną na płaszczyźnie. \pause 

\bigskip
Rozważmy teraz krzywą $\beta\colon \R\to \R^2\subset \R^3$ zadaną wzorem
\mode<article>{\[\beta(t)=(t^3,t^3,t^3).\]} 
\mode<presentation>{\[\beta(t)=(t^{\,3},t^{\,3},t^{\,3}).\]}Obraz tej krzywej 
jest oczywiście prostą, jednak podobne przeliczenia jak powyżej pokazują, że 
\[\frac{\widehat{D}\beta'(t)}{dt}=\frac{d\beta'(t)}{dt}=(6t,6t,6t)\neq 0.\]
\end{uwaga}

\end{frame}
%%%%%%next-slide%%%%%
\begin{frame}
\mode<article>{\begin{przyklad}}
Pokażemy, że każdy wielki okrąg na jednostkowej sferze $S^2\subset \R^3$ jest obrazem pewnej krzywej geodezyjnej. 

Ze wzgędu na symetrię sfery wystarczy pokazać, że równik $S^2$ sparametryzowany jako $\gamma\colon (-\pi,\pi)\to S^2$ wzorem \[\gamma(t)=(\cos t,\sin t,0)\] jest geodezyjną. \pause Styczne pole wektorowe jest równe \[\gamma'(t)=(-\sin t, \cos t, 0),\] wobec czego
\mode<presentation>{\begin{multline*}
\frac{\widehat{D}\gamma'(t)}{dt}=\Pi_{T_{\gamma(t)}S^2}\left(\frac{d\gamma'(t)}{dt}\right)=\\\pause
=\Pi_{T_{\gamma(t)}S^2}\left((-\cos(t),-\sin(t),0)\right)
=\Pi_{T_{\gamma(t)}S^2}(-\gamma(t))=0.
\end{multline*}}
\mode<article>{\begin{multline*}
\frac{\widehat{D}\gamma'(t)}{dt}=
\Pi_{T_{\gamma(t)}S^2}\left(\frac{d\gamma'(t)}{dt}\right)=
\Pi_{T_{\gamma(t)}S^2}\left((-\cos(t),-\sin(t),0)\right)=\\\pause
=\Pi_{T_{\gamma(t)}S^2}(-\gamma(t))=0,
\end{multline*}
ponieważ sfera ma tę własność, że przestrzeń styczna w każdym punkcie jest prostopadła do prostej wyznaczonej przez ten punkt i środek sfery.}
\mode<article>{\end{przyklad}}
\end{frame}

Następujący lemat pokazuje związki pomiędzy geodezyjnymi a ich reparametryzacjami.
%%%%%%next-slide%%%%%
\begin{frame}
\begin{lemat}\label{lem:param-of-geodesics}
Niech $M\subset \R^3$ będzie powierzchnią i niech $\gamma\colon(a,b)\to M$ będzie nie-stałą geodezyjną. Wtedy następujące twierdzenia są prawdziwe.
\begin{enumerate}
\item Krzywa $\gamma$ ma stałą (nie-zerową) prędkość (tj. $||\gamma'(t)||=c\neq 0$ dla wszysktich $t\in(a,b)$.
\pause\item Reparametryzacja $\widetilde{\gamma}=\gamma\circ h\colon (c,d)\to (a,b)\to M$ jest krzywą geodezyjną wtedy i tylko wtedy gdy $h$ jest funkcją afiniczną.
\pause\item Jeśli obraz krzywej $\delta\colon (c,d)\to M$ jest zawarty w obrazie krzywej $\gamma$, $\delta(c,d)\subset \gamma(a,b)$, wtedy $\delta$ jest geodezyjną wtedy i tylko wtedy gdy ma stałą prędkość.
\end{enumerate}
\end{lemat}

\pause \textcolor{ared}{\textbf{Dowód: }}\\
\mode<article>{Dowody pierwszych dwóch punktów są łatwe (można potraktować je jako ćwiczenia na zrozumienie definicji geodezyjnej). Dowód trzeciego jest bardziej wymagający.}
Pomijamy.
\hfill $\square$

\end{frame}

\mode<all> 
